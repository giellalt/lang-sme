\documentclass[a4paper,nynorsk]{article}
\usepackage{babel}
\usepackage{ucs}
\usepackage[utf8x]{inputenc}
\usepackage[T1]{fontenc}
\usepackage{a4wide}

\usepackage{hyperref}
\usepackage{graphics}
\usepackage{url}
\usepackage{covington}

\begin{document}


\title{Disambiguering av homonymi i nord- og lulesamisk}

\author{Trond Trosterud og Linda Wiechetek\\
Det humanistiske fakultet\\
Universitetet i Tromsø}

\maketitle

\tableofcontents %%%



% Plan:
% 
% Vise homonymien i sme/smj (utdrag)
% Vise skilnad i forhold til homonymien
% Vise metodologien - formalismen? eller framgangsmåte?
% Skilje mellom homonymi i ulik grammatisk kontekst og homonymi i lik grammatisk kontekst i forhold til enkelte problemer !!! eg syns du burde definere "lik"/"ulik" grammatisk kontekst
% Ulik kontekst: t.d. Infinitiv vs. 1pl (sme) gsg vs npl (smj)
% Lik kontekst: t.d. logpl/comsg
% Vise reglar for begge typar
% Drøfte skilnader mellom sme og smj - i forhold til disambigueringa? 
% Konklusjon

\section{Innleiing} 

%%%

I nordsamisk løpande tekst har kvar ordform i gjennomsnitt 2,6 moglege grammatiske analyser, og i lulesamisk har kvar ordform 2,0 moglege analyser.\footnote{Vi vil takke Marit Julien for det arbeidet ho har gjort med utarbeidinga av regelsettet for nordsamisk. Vi vil også takke kollegene våre i Giellatekno- og Divvun-prosjekta for det kollektive arbeidet vi alle er ein del av: Børre Gaup, Saara Huhmarniemi, Ilona Kivinen, Sjur Moshagen, Thomas Omma, Maaren Palismaa og Tomi Pieski. Den som gjorde arbeidet med samisk språkteknologi mogleg var likevel Pekka Sammallahti, som tidleg på nittitalet skreiv grunnlaget til ein morfologisk analysator for nordsamisk.}. Grunnen til at nordsamisk har meir homonymi enn lulesamisk er framfor alt at lulesamisk, fonologisk og morfologisk sett, er eit meir konservativt språk enn nordsamisk. Der nordsamisk final \emph{-p, -t, -k} har falle saman i \emph{-t}, og akkusativ og genitiv er identisk (bortsett frå i talord), har lulesamisk halde oppe desse distinksjonane. Den lulesamiske systematiske homonymien genitiv sg = nominativ pl er ikkje like omfattande som dei nordsamiske. %%%

I begge språka er lokativ/inessiv pl identisk med komitativ sg. Tabellane \ref{smecas} og \ref{smjcas} viser bøyingsmønsteret for det samiske substantivet \emph{giella} ''språk''.%%%

\begin{table}[htdp]
\caption{Nordsamisk kasusparadigme}
\begin{center}
\begin{tabular}{|l|c|c|}
\hline
Kasus  & Sg & Pl \\
\hline
Nom & giella & gielat \\
\hline
Acc & giela & gielaid \\\cline{1-1}
Gen & giela & gielaid \\
\hline
Ill & gillii & gielaide \\
\hline
Loc & gielas & gielain \\
\hline
Com & gielain & gielaiguin \\
\hline
Ess & \multicolumn{2}{c|}{giellan}  \\
\hline
\end{tabular}
\end{center}
\label{smecas}
\end{table}%

\begin{table}[htdp]
\caption{Lulesamisk kasusparadigme}
\begin{center}
\begin{tabular}{|l|c|c|}
\hline
Kasus  & Sg & Pl \\
\hline
Nom & giella & giela \\
\hline
Acc & gielav & gielajt \\
\hline
Gen & giela & gielaj \\
\hline
Ill & giellaj & gielajda \\
\hline
Ine & gielan & gielajn \\
\hline
Ela & gielas & gielajs \\
\hline
Com & gielajn & gielaj \\
\hline
Ess & \multicolumn{2}{c|}{giellan}  \\
\hline
\end{tabular}
\end{center}
\label{smjcas}
\end{table}%

I tillegg til homonymiane som er vist her er det også ei relativt lita gruppe ord, både i nord- og lulesamisk, som ikkje har stadieveklsing, eller der stadievekslinga er synleg berre i tale, og ikkje i skrift. For desse orda blir homonymifelta større, t.d. blir nominativ = genitiv = akkusativ i nordsamisk for  orda i figur \ref{nomgenlist}, som har fått eit eige sett, eit langt frå komplett sett (nokre av desse orda kan også ha stadieveksling):%%%

\begin{figure}[htbp]
\begin{center}
\begin{verbatim}
LIST NOM-GEN-NOUN = "ákšu" "vahkku" "plána" "ulbmil" "lága" "journála" "stáhta" "jolá" ;
\end{verbatim}
\caption{Sett av ord i nordsamisk som ikkje har stadieveksling}
\label{nomgenlist}
\end{center}
\end{figure}

Når det gjeld verba (tabell \ref{verb}), har nordsamisk ein systematisk homonymi mellom infinitiv og 1. person pluralis presens. For trestava verb representer den same forma også 3. person pluralis og 2. person singularis preteritum, alle er \emph{muitalit}. For lulesamisk er dette tre ulike former: Infinitiven er \emph{mujttalit}, 1. person pluralis har halde på opprinneleg \emph{-p} og er \emph{mujttalip}, mens dei to andre grammatiske orda har same form, \emph{mujttali}.%%%

\begin{table}[htdp]
\caption{Nord- og lulesamiske verb i indikativ}
\begin{center}
\begin{tabular}{|l|c|c||c|c|}
\hline
Person & \multicolumn{2}{|c||}{Nordsamisk \textit{muitalit}} & \multicolumn{2}{|c|}{Lulesamisk \textit{mujttalit}} \\
\hline
 & Presens & Preteritum & Presens & Preteritum \\
\hline \hline 
Sg1 & muitalan & muitalin &  mujttalav & mujttaliv \\
\hline
Sg2 & muitalat & muitalit & mujttala & mujttali \\
\hline
Sg3 & muitala & muitalii & mujttal & mujttalij \\
\hline
\hline
Du1  & muitaletne & muitaleimme & mujttalin & mujttalijma \\
\hline
Du2 & muitaleahppi & muitaleidde & mujttalihppe & mujttalijda \\
\hline
Du3 & muitaleaba & muitaleigga & mujttalibá & mujttalijga \\
\hline
\hline
Pl1  & muitalit & muitaleimmet & mujttalip & mujttalijma \\
\hline
Pl2 & muitalehpet & muitaleiddet & mujttalihpit & mujttalijda  \\
\hline
Pl3 & muitalit & muitaledje & mujttali & mujttalin \\
\hline
\hline
\end{tabular}
\end{center}
\label{verb}
\end{table}%

Når vi høyrer samisk tale er vi nesten aldri i tvil om kva form talaren har meint. Grunnen til det er at orda aldri blir ytra i isolasjon, men i ein kontekst som gjer at berre ei tolking er mogleg. Eg vil gå inn på måtar å formalisere desse kontekstane på. Ein sentral konklusjon vil bli at formaliseringa kan bli reint grammatisk berre til eit visst punkt, og at vi deretter er avhengig av leksikalsk kontekst for å kunne disambiguere rett. Desse resultata er relevante for lingvistiske modellar som hevdar å kunne modellere dei kognitive prosessane som trengst for å forstå språklege ytringar.%%%
% De fins sammenhenger der vi faktisk møter ekte ambiguitet likevel og der heller ikkje vedkommende kan være sikker på hva som menes. To eksempler:
%oktiigullevašvuođa sámi servodagain
%1. samhørighet/solidaritet med det samiske samfunnet 2.solidaritet i de samiske samfunnene
%mo aššiin manná
%1. korleis går det med saka 2. korleis går det i sakene 

\section{Homonymi i løpande tekst}

%Ein ting er homonymimønsteret i morfologien, ein annan ting er homonymimønsteret i løpande tekst. Nokre former er meir i bruk enn andre.%%%

\subsection{Lulesamisk} 
 
Ei analyse av eit lulesamisk korpus på 145491 ord (det lulesamiske nytestamentet) gjev 280200  analyser, eller 1,92 analyser per ordform. Jf. tabell \ref{smjhom} for eit oversyn over fordelinga av homonymien.%%%

Vanlegast er homonymien mellom eintal genitiv og fleirtal nominativ, som vi såg i tabell \ref{smjcas}. Deretter kjem eit sett av verbformer, 2815 av dei 2867 tilfella er copula \emph{le}. Det neste tilfellet har med namneformatering å gjere. Vi har kasuskongruens mellom ein del fleirtalsformer, genitiv, komitativ, og, for visse stammeklasser, illativ. Interrogative og relative pronomen er identiske, dette er eit tilfelle av funksjonsdeling og ikkje eigentleg homonymi. Personlege pronomen skil ikkje mellom genitiv og akkusativ. Blant verba er Pl3 for visse stammeklasser identisk med første person dualis, for andre med preteritum andre person singularis.%%%

\begin{table}[htdp]
\caption{Dei ti vanlegaste homonymiklassene i lulesamisk løpande tekst}
\begin{center}
\begin{tabular}{|c|r|l|}
\hline
\% hom & # hom & homonymiklasser \\
\hline
4,21\% & 6125 & N Sg Gen  =  N Pl Nom  \\
1,97\% & 2867 & V Ind Prs Sg2  =  V Ind Prs ConNeg  =  V Ind Prs Sg3  =  V Ind Prs Pl3  \\
1,37\% & 1995 & N Prop Mal Sg Nom  =  N Prop Mal Sg Attr  \\
1,20\% & 1746 & N Pl Gen  =  N Pl Com  \\
1,14\% & 1653 & Pron Interr Sg Nom  =  Pron Rel Sg Nom  \\
1,13\% & 1640 & Pron Pers Sg3 Gen  =  Pron Pers Sg3 Acc  \\
1,09\% & 1585 & V TV Ind Prt Pl3  =  V TV Ind Prs Du1  \\
1,06\% & 1543 & V TV Ind Prs Pl3  =  V TV Ind Prt Sg2  \\
0,95\% & 1385 & N Pl Gen  =  N Sg Ill  =  N Pl Com  \\
0,90\% & 1312 & N Sg Ela  =  N Sg Acc PxSg3  =  N Sg Gen PxSg3  \\
\hline
\end{tabular}
\end{center}
\label{smjhom}
\end{table}%

\subsection{Nordsamisk}

Eit tilsvarande nordsamisk korpus (nytestamentet) på 165885 ordformer gjev 389524 analyser, eller 2,35 analyser per ordform, jf. tabell \ref{smehom}.%%%

Den vanlegaste homonymien er den mellom akkusativ og genitiv, desse kasusa er alltid identiske. Viss vi ignorerer fleirbruksfunksjonen til interrogative og relative pronomen, er dei første tre plassane på tabellen oppteken av akkusativ/genitiv-homonymi, meir enn ein tidel av homonymitilfella er av denne typen. Andre store grupper er lokativ \emph{-s}, som er identisk med possessivsuffikset i genitiv og akkusativ, og infinitiv og presens første person fleirtal, som alltid er identiske. Legg merke til homonymien mellom fleirtal lokativ og eintal komitativ, som kjem som nr. 20 på lista, med 0,68 \% av tilfella.%%%


\begin{table}[htdp]
\caption{Dei tjue vanlegaste homonymiklassene i nordsamisk løpande tekst}
\begin{center}
\begin{tabular}{|r|r|l|}
\hline
\% hom & \# hom & honomnymiklasser \\
\hline
5,27\% & 7670 & N Sg Acc /  N Sg Gen \\
2,27\% & 3308 & N Sg Nom /  N Sg Acc /  N Sg Gen \\
1,82\% & 2645 & Pron Interr Sg Nom /  Pron Rel Sg Nom \\
1,78\% & 2583 & N Pl Gen /  N Pl Acc \\
1,73\% & 2516 & Pron Pers Sg3 Nom /  Pcle \\
1,62\% & 2363 & N Prop Mal Sg Nom /  N Prop Mal Sg Attr \\
1,56\% & 2275 & N Sg Loc /  N Sg Acc PxSg3 /  N Sg Gen PxSg3 \\
1,53\% & 2228 & Pcle /  CS \\
1,34\% & 1955 & V TV Ind Prs Sg3 /  V TV VGen /  V TV Ind Prs ConNeg /  \\  
       &      & V TV Imprt Prs Sg2 /  V TV Imprt Prs ConNeg \\
1,26\% & 1840 & V IV Inf /  V IV Ind Prs Sg2 /  V IV Ind Prs Pl1 / \\
       &      & V IV Ind Prs ConNeg /  V IV Ind Prs Pl3 \\
1,23\% & 1789 & CS /  Adv \\
1,21\% & 1758 & Pron Pers Sg3 Gen /  Pron Pers Sg3 Acc \\
1,20\% & 1745 & V IV Ind Prt Pl3 /  V IV Ind Prs Du1 \\
1,10\% & 1603 & CS /  CC \\
1,09\% & 1593 & Pron Dem Sg Gen /  Pron Dem Sg Acc /  Pron Pers Sg3 Gen / \\
       &      & Pron Pers Sg3 Acc \\
0,83\% & 1210 & V TV Ind Prt Pl3 /  V TV Ind Prs Du1 \\
0,79\% & 1155 & Pron Dem Sg Gen /  Pron Dem Sg Acc \\
0,79\% & 1155 & N Prop Mal Sg Gen /  N Prop Mal Sg Acc \\
0,78\% & 1128 & Pron Pers Sg3 Nom /  Pcle /  Pron Dem Sg Nom / \\
       &      & Pron Pers Pl3 Nom /  Pron Dem Pl Nom \\
0,68\% &  983 & N Pl Loc /  N Sg Com \\
\hline
\end{tabular}
\end{center}
\label{smehom}
\end{table}%


\section{Allment om disambiguering}

Når vi skal disambiguere står vi ovafor i prinsippet to ulike slags tilfelle: Tilfella der dei homonyme formene opptrer i ulik grammatiske kontekst, og tilfelle der dei opptrer i lik grammatiske kontekst. Medlemmane i same homonymiklasse kan opptre både i like og i ulike grammatiske kontekstar. Genitiv og akkusativ er døme på dette: Som utvetydig objekt er det akkusativ, som utvetydig postposisjonskomplement er det genitiv, og som tidsadverbial kan det vere begge delar. Dessutan kan det vere tilfelle der elles distinkte kontekstar er nøytralisert: I ein streng \textit{TV X Y Z Po}, eller den samiske versjonen av \textit{Eg ser Pers bils dørs bak} kan vi ikkje vite om eg ser Per bak bildøra, eller Pers bil bak døra, dvs. vi veit ikkje kor akkusativen skal vere (på X eller Y), det kan til og med vere at setninga ikkje har akkusativ i det heile, at eg driv og ser meg rundt attom bildøra til Per.%%%


\subsection{Ulike innfallsvinklar til parsing}

Det finst mange innfallsvinklar til disambiguering av homonymi i tekst. Dei to viktigaste er statistisk og grammatisk disambiguering. Statistisk basert disambiguering tar utgangspunkt i eit korpus av manuelt tagga tekst. Taggaren lærer mønster av dette korpuset, og går ut i frå at ny tekst oppfører seg på same måten. Ulempa med statistisk disambiguering er at dei ser ut til å ha eit tak på i underkant av 97 \% korrekt resultat.  Grammatisk baserte innfallsvinklar kjem i to versjonar, ovanfrå og ned, eller nedanfrå og opp. Ovanfrå-og-ned-taggarar (typisk: LFG-taggarar) prøver ut hypotesar om kva syntaktisk tre setninga kan representere. Viss taggaren klarer det, er resultatet svært godt: ein syntaktisk frasestruktur med opplysningar om form, funksjon og hierarkisk struktur. Problemet med slike taggarar er at dei er både for gode og for dårlege: Same setning kan representerast av mange ulike hierarkiske analyser, og ei og same setning kan dermed fåhundrevis, ja tusenvis av ulike analyseframlegg. På den andre sida er naturleg språk ofte ikkje i samsvar med syntaktiske reglar, talararendrar meining midt i setninga, dei føyer inn lange digresjonar, eller setninga kan rett og slett bli for kompleks. I slike tilfelle bryt ovanfrå-og-ned-parsarar saman, og er ikkje i stand til å generere ein S-node i det heile. Sjølv gode parsarar av denne typen klarer sjeldan å analysere meir enn 60 \% av setningane i løpande tekst.%%%

Den innfallsvinkelen vi arbeider etter er ein grammatisk basert parser nedanfrå-og-opp-parsar. og dermed ikkje bunde av den øvre grensa på 97 \%. På den andre sida er han, på same måten som statistiske parsarar, ein nedanfrå-og-opp-parsar, og dermed robust: Også setningsfragment og komplekse setningar blir analysert. Det finst ulike typar slike parsarar, vår modell baserer seg på føringsgrammatikk, eller constraint grammar, og ser grammatikken som eit sett føringar, eller constraints, på kva kontekstar dei ulike morfologiske analysene kan opptre i.%%%

\subsection{Formalisme}

Regelformatet til føringsgrammatikken er relativt enkelt, reglane er skrive som ein $OPERASJON$ på eit mål i ein gjeve kontekst, som vist i eksempel \ref{format}. %%%

\begin{example}\label{format}
OPERASJON mål IF kontekst ;
\end{example}

Reglane virkar på ei og ei ordform, som er definert som posisjon \emph{0} av regelen. Ordet til venstre er \emph{-1}, to ord til venstre er \emph{2}, og to eller færre ord til venstre er \emph{*-2}- Regelformatet inneheld operasjonane \emph{ADD, MAP, REMOVE, SELECT}, dessutan kontekstidentifikatoren \emph{IF} og vilkårsoperatorane \emph{BARRIER, LINK}. Reglane med \emph{ADD} og \emph{MAP} har også ein eigen målidentifikator \emph{TARGET}.%%%

\begin{table}[htdp]
\caption{Regelformatet i føringsgrammatikken}
\begin{center}
\begin{tabular}{|c|l|}
\hline
 ADD & legg til ny tagg \\
 MAP & legg til tagg \\
 REMOVE & fjern lesing \\
 SELECT & vel lesing \\
\hline 
 IF & introduserer kontekstavgrensande operatorar \\
\hline 
 0 & posisjonen til målet for operatorane \\
 -1 & ein posisjon til venstre \\
 $\ast2$ & to eller fleire posisjonar til høgre \\
\hline 
 BARRIER & stoppar skanninga \\
 LINK & fører skanninga vidare til eit nytt vilkår \\
\hline 
\end{tabular}
\end{center}
\label{default}
\end{table}%



$SELECT$ og $REMOVE$ er dei sentrale operatorane, dei vel og fjerner lesingar, og er inverse av kvarandre: $SELECT$ A inneber at alle andre lesingar enn A blir fjerna. $MAP$ og $ADD$ er operasjonar som legg til nye taggar, i vårt tilfelle legg dei til syntaktiske taggar, basert på plasseringa konstituentane har i setninga. $BARRIER$ blir brukt i lag med *-operatoren. Når disambigueraren skannar setninga etter eit vilkår (t.d. \textit{*-1 B}, eller ''sjå etter $B$ til venstre''), i \ref{ped}, kan vi legge inn ei mogleg barriere  (\textit{*-1 B BARRIER C} ''... med mindre du finn $C$ før du kjem så langt). Vilkårsoperatoren $LINK$ fører oss vidare frå eit mål: \textit{(IF *-1 (N Nom) LINK *-1 CC LINK -1 (N Nom)} ber oss om å leite etter eit substantiv til venstre, med ein konjunksjon ein stad til venstre for seg, som på si side att har eit nytt sybstantiv umiddelbart til venstre for seg att.%%%

\begin{example}\label{ped}
\item[(a)] SELECT A IF *-1 B
\item[(b)] SELECT A IF *-1 B BARRIER C
\item[(c)] SELECT V-PL IF *-1 (N Nom) LINK *-1 CC LINK -1 (N Nom))
\end{example}

Eit døme på ein regel er \ref{regeleks}, ein regel som vel lesinga (dvs. analysen) \emph{Inf} viss vi eitt ord til venstre finn eit medlem av settet \emph{VFIN}, som  tidlegare er definert som settet av \emph{V-MOOD} minus alle analyser som inneheld negasjonsforma \emph{ConNeg}. Cf. \ref{regeleks}.  %%%

\begin{example}\label{regeleks}
\item[(a)] SET V-MOOD = Ind | Pot | Imprt | ImprtII | Cond ; 
\item[(b)] SET VFIN  =  V-MOOD - ConNeg ; 
\item[(c)] SELECT Inf IF (-1 VFIN); 
\end{example}%

\subsection{Hierarkisk struktur i ein lineær modell}

Eit problem med disambigueringa er å klare å estimere frasestrukturen i setninga, når formalismen berre ser ein streng. Vi kan t.d. ikkje seie \textit{substantiv til venstre for konjunksjon til venstre for NP}, når analyseprogrammet berre ser ein streng, og ikkje ein frase (NP). %%%

For å få til å operere med hierarkiske strukturar i ein flat streng generaliserer vi over delar av strengen. For å simulere NP-ar definerer vi t.d. eit komplementært sett $NPNH$, der NPNH er står for ''NOT-PRE-NP-HEAD'', ved først å sjå på kva som kan stå prenominalt i NP, og så definere komplementet til dette settet, som ($WORD - PRE-NP-HEAD$, der $WORD$ er definert som settet av alle ord). %%%

\begin{example}
SET PRE-NP-HEAD = (Prop Attr) | (Prop @PROP>) | A | (Pron Pers Gen) | (N Gen) | 
                  Num | Cmpnd | CC | (Pron Dem)  | (Pron Refl Gen) | 
                  (Pron Indef) | (PrfPrc @AN>) |  (PrfPrc @PrcN>) | PrsPrc ; \\
SET NPNH        = WORD - PRE-NP-HEAD | ABBR ; \\                 
\end{example}

Det er meir å seie om NP-struktur enn dette, det finst relativsetninger og andre komplekse apposisjoner og modifikatorer som konkurrerer med adverbial og objekt, men med $NPNH$-definisjonen dekkar vi ein svært stor del av alle tilfella. %%%

\subsection{Setning eller avsnitt som grense}

Disambigueringsprogrammet skanner strengen mellom to grensemerke, desse er definert av brukaren. Definisjonen i \ref{delimiters} a viser grensene våre for disambiguering av setningar. I tillegg til det har vi ein separat disambigueringsmodul som har avsnitt heller enn setning som domene, og som har eit separat regelsett. Dette gjør det mulig å inkludere kontekst- og diskurs informasjon som kjem før sjølve setninga. I preprosesseringa av teksten legg vil til symbolet ¶ attom kvart avsnitt. Avsnittsdisambigueraren blir brukt til å disambiguere homonymi vi ikkje har vorte kvitt med setningsdisambigueraren, der det kan hjelpe å sjå ut over eiga setning, som td.d. i pro-drop-tilfelle, der vi kan identifisere personen til det finitte verbet via ein passande antesedent i forrige setning. %%%


\begin{example}\label{delimiters}
\begin{itemize}
\item[(a)] {DELIMITERS = ''<.>'' ''<!>'' ''<?>'' ''<...>'' ;}
\item[(b)] {DELIMITERS = ''<¶>'';}
\end{itemize}
\end{example}

I prinsippet er det mogleg å ha eit både større og mindre vindauge enn dette, men vi opererer altså med setning og avsnitt. %%%


\subsection{Status quo}

Vi har på \textit{giellatekno}-prosjektet laga ein disambiguerar for nordsamisk, som består av 2055 reglar, og som reduserer graden av homonymi frå 2,35  til 1,027 per ordform. For lulesamisk har vi ein prøveversjon av disambigueraren, med 154 reglar, den reduserer graden av homonymi frå 1,92 til 1,45, altså til det halve. %%%

\begin{table}[htdp]
\caption{Homonymi utan (a/o) og med (d/o) disamnbiguering}
\begin{center}
\begin{tabular}{|l|r|r|r|r|r|}
\hline
 & ord & analyser & a/o & disamb & d/o \\
\hline
nordsamisk & 165885 & 389524 & 2,35 & 170401 & 1,02 \\
lulesamisk &  145491 & 280200 &  1,92 & 211316 & 1,45 \\
\hline
\end{tabular}
\end{center}
\label{posthom}
\end{table}%

Dette er ikkje like godt som for dei beste føringsgrammatikkane, men skilnaden er ikkje stor. %%%


\section{Disambiguere grammatisk skilde kontekstar}



Vi ser no på disambiguering der dei homonyme formene opptrer i kontekstar som det er mogleg å skilje frå kvarandre på grammatisk grunnlag. %%%

\subsection{Verbperson}

I nordsamisk er første person pluralis alltid identisk med infinitiv, som vi såg i tabell \ref{verb}. Ordforma \emph{vuolgit} kan bety både ‘‘dra’’ og ‘‘eg dreg’’. I eksempel \ref{mifevutxt} ser vi ei enkel setning, \emph{Mii fertet vuolgit} ‘‘Vi må dra’’,  der begge verbformene kan vere både infinitiv og første person pluralis, men der disambigueraren er i stand til å skilje mellom dei. %%%

\begin{example}\label{mifevutxt}
\gll Mii fertet vuolgit.
 Vi må dra.
\gln 
\glend
\end{example}


Den morfologiske analysatoren gjev oss alle dei moglege analysene, i figur \ref{mifevumulti}.%%%

\begin{figure}[htbp]
\begin{center}
\begin{verbatim}
"<Mii>"
         "M" Num Ill
         "mii" Pron Interr Sg Nom
         "mii" Pron Rel Sg Nom
         "mun" Pron Pers Pl1 Nom
"<fertet>"
         "fertet" V IV Ind Prs Sg2
         "fertet" V IV Ind Prs Pl1
         "fertet" V IV Inf
"<vuolgit>"
         "vuolgit" V* IV N Actor Pl Nom
         "vuolgit" V IV Inf
         "vuolgit" V IV Ind Prs Pl1
"<.>"
         "." CLB
\end{verbatim}
\caption{Morfologisk analyse av \emph{Mii fertet vuolgit}}
\label{mifevumulti}
\end{center}
\end{figure}

Disambiguatoren fjerner dei irrelevante analysene, og returnerer analysen i figur \ref{mifevu}. %%%

\begin{figure}[htbp]
\begin{center}
\begin{verbatim}
"<Mii>"
        "mun" Pron Pers Pl1 Nom @SUBJ
"<fertet>"
        "fertet" V IV Ind Prs Pl1 @+FAUXV
"<vuolgit>"
        "vuolgit" V IV Inf @-FMAINV
"<.>"
        "." CLB <<<
\end{verbatim}        
\caption{Analyse av den nordsamiske setninga \textit{Mii fertet vuolgit}}
\label{mifevu}
\end{center}
\end{figure}

Regelen som vel Pl1-lesinga av verbet \emph{fertet} er vist i \ref{mifevurle}. Regelen ser etter eit mogleg pronomen til venstre (\textit{*-1 MII-PERS}), i leitinga etter dette pronomenet får vi ikkje møte på finitte verb eller skiljeteikn på vegen, og heller ikkje eit nytt verb i første person pluralis til venstre for pronomenet:%%%

\begin{example}\label{mifevurle}
SELECT (V Pl1) IF (*-1 MII-PERS BARRIER VFIN OR PUNCT LINK NOT *-1 V-PL1 
	BARRIER NOT-ADV-PCLE OR CLB);
\end{example}	

Regelen som vel infinitiv for \emph{vuolgit} (nummer \ref{infverbregel}) plukkar ut infinitivslesinga viss det står eit verb som tar infinitivskomplement til venstre for seg, (\textit{*-1 INF-VERB}). %%%

\begin{example}\label{infverbregel}
SELECT Inf IF (*-1 INF-VERB)(NOT *1 Inf BARRIER VFIN OR CS OR CP);
\end{example}

Verbet \emph{fertet} ‘‘måtte’’ er eit slikt verb, eitt blant mange, som vi kan sjå i settet $INF-VERB$ i figur \ref{infverbsett}. %%%


\begin{figure}[htbp]
\begin{center}
\begin{verbatim}
LIST INF-VERB = "adnot" "astat" "ádjánit" "áigut" "álgit" "ásahit" "bágget" 
		"bávččagit" "beassat" "berret" "bivdit" "bivvat" "bistit" 
		"boahtit" "bovdet" "čohkkát" "čohkkedit" "čohkkánit" "čuoččahit" 
		"čuoččastit" "čuorvut" "čurggodit" ("dadjat" Pass) "dagahit" 
		"dáhttut" "dáidit" "dárbbašit" "diktit" "doaivut" "doapmat" 
		"duostat" "fertet" "fuobmát" "fuolahit" "galgat" "gáibidit" 
		"gárrut" "gártat" "geahččalit" "geatnegahttit" "gierdat" "gillet" 
		"gohččut" "háhppehit" "hálidit" "háliidit" "hárjánit" "heivet" 
		"lávet" "mannat" "máhttit" "máššat" "movttiidahttit" "muitit" 
		"nagodit" "navdit" "oahpahit" "oahppat" "oažžut" "ollet" "ribahit" 
		"riepmat" "sáhttit" "seahtit" "sihtat" "soaitit" "suovvat" "šaddat" 
		"stađđat" "veadjit" "viggat" "viššat" "vuogáiduvvat" "vuolgit" 
		"vuollánit" "vuordit";
\end{verbatim}
\caption{Settet INF-VERB}
\label{infverbsett}
\end{center}
\end{figure}

Subjektet kan sjølvsagt vere meir komplekst, i og med at vi analyserer autentiske setningar har vi ofte svært kompliserte konstruksjonar. Eksempel \ref{mjmeks} viser eit litt meir komplisert subjekt, koordinert personleg pronomen og NP: %%%

\begin{example}\label{mjmeks}
\gll Mun ja Maria váhnemat leat boahtán.
     Eg  og Marias foreldre er  komne.
\gln     
\glend
\end{example}%


Setninga i \ref{mjmeks} får analysen i figur \ref{mjam}. %%%

\begin{figure}[htbp]
\begin{center}
\begin{verbatim}
"<Mun>"
        "mun" Pron Pers Sg1 Nom @SUBJ
"<ja>"
        "ja" CC @CC
"<Maria>"
        "Maria" N Prop Fem Sg Gen @GN>
"<váhnemat>"
        "váhnen" N Pl Nom @SUBJ
"<leat>"
        "leat" V IV Ind Prs Pl1 @+FAUXV
"<boahtán>"
        "boahtit" V IV PrfPrc @-FMAINV
"<.>"
        "." CLB <<<
\end{verbatim}
\caption{Analyse av \textit{Mun ja Maria váhnemat leat boahtán}}
\label{mjam}
\end{center}
\end{figure}

 
Regelen som gjev oss Pl1 for det finitte verbet er regel  \ref{koordinertsubj}. %%%
 
\begin{example}\label{koordinertsubj}
SELECT (V Pl1) IF (*-1 Nom BARRIER NOT-ADV-PCLE LINK *-1 CC BARRIER NPNH
	LINK -1 (Pers Sg1 Nom) OR (Pers Du1 Nom) OR (Pers Pl1 Nom));
\end{example}%

Regel \ref{koordinertsubj} vel Pl1 viss det til venstre finst ein nominativ, utan at det er noko anna enn adverb eller partikkel imellom, slik at denne nominativen er lenka til ein konjunksjon foran NPen (derfor BARRIER NPNH), til venstre for konjunksjonen skal vi finne eit pronomen i nominativ. %%%

%Vi kjem relativt langt med disambigueringa av infinitivar og første person pluralis av verb, grunnen til det er at dei ulike kontekstane skil seg frå kvarandre, grammatisk sett. Dei problematiske tilfella som står att involverer ofte Pl1 vs. Pl3, infinitiv klarer vi som regel å disambiguere. Før disambiguering er homonymimønster som involverer infinitiv involvert i 6,94 \% av alle homonymitilfella. %%%




\subsection{Genitiv vs. akkusativ i nordsamisk}

Genitiv og akkusativ opptrer ofte i skilde kontekstar. For at vi skal ha akkusativ må vi enten ha eit tidsadverbial eller eit transitivt verb. Vi har gått gjennom alle verba i det nordsamiske leksikonet vårt og merka dei for transitivitet. Det er sjølvsagt mogleg å bruke eit transitivt  verb intransitivt, som t.d. den samiske omsetjinga av \textit{Et du hos Per}, der \textit{Per} ikkje skal ha akkusativ, trass i at det er komplementet til eit transitivt verb. %%%

Komplementet til ein postposisjon skal vere i genitiv. Possessoren til eit substantiv skal også vere i genitiv, denne possessoren kan stå til både eit postposisjonskomplement og eit objekt. Dermed kan vi ha både $VGGGP$ (med intransitiv bruk av verbet, $VAGGP$, $VGAGP$ og (i tilfelle postposisjonen er eit adverbial) $VGGAP$. %%

Det hendar strengen av substantiv gjev vink om den rette løysinga, t.d. viss eit av substantiva er eit eigennamn er det sannsynlegvis possessor heller enn det er modifisert av eit fellesnamn. Jfr. \ref{hund}. %%%

\begin{example}\label{hund}
\gll Mun oainnán beatnaga Máreha uvssa duohkán.
 Eg ser hund.GA Marit.GA dør.GA bak
\glt 'Eg ser ( ein hund bak Marits dør / ??hundens Marit bak døra)'
\glend
\gll Mun oainnán beatnaga nieidda uvssa duohkán.
  Eg ser hund.GA jente.GA dør.GA bak
\glt 'Eg ser ( ein hund bak jentas dør / hundens jente bak døra )'
\glend
\end{example}

Den første setninga klarer disambigueraren vår å disambiguere, jf. figur \ref{Mareha}. %%%

\begin{figure}[htbp]
\begin{center}
\begin{verbatim}
"<Mun>"
        "mun" Pron Pers Sg1 Nom @SUBJ
"<oainnán>"
        "oaidnit" V TV Ind Prs Sg1 @+FMAINV
"<beatnaga>"
        "beana" N Sg Acc @OBJ
"<Máreha>"
        "Máret" N Prop Fem Sg Gen @GN>
"<uvssa>"
        "uksa" N Sg Gen @GN>
"<duohkán>"
        "duohki" N Sg Acc PxSg1 @OPRED
"<.>"
        "." CLB <<<
\end{verbatim}
\caption{Analyse av den nordsamiske setninga \textit{Mun oainnán beatnaga Máreha uvssa duohkán.}}
\label{Mareha}
\end{center}
\end{figure}


Det som gjev oss akkusativ for \textit{beatnaga} er nett det faktum at neste ord er eit eigennamn, jf. \ref{propgen}: %%%

\begin{example}\label{propgen}
SELECT Acc IF (*-1 V-TRANS-ACT BARRIER NOT-ADV-PCLE LINK NOT *-1 (@OBJ) BARRIER CS)
	(NOT -1 Gen)(0C N)(1 (Prop Gen));
\end{example}

I det andre tilfellet har vi ikkje nokon hjelp, og vi får feil resultat, som vist i figur \ref{beatnaga}. %%%


\begin{figure}[htbp]
\begin{center}
\begin{verbatim}
"<Mun>"
        "mun" Pron Pers Sg1 Nom @SUBJ
"<oainnán>"
        "oaidnit" V TV Ind Prs Sg1 @+FMAINV
"<beatnaga>"
        "beana" N Sg Gen @GN>
"<nieidda>"
        "nieida" N Sg Gen @GN>
"<uvssa>"
        "uksa" N Sg Gen @GN>
"<duohkán>"
        "duohki" N Sg Acc PxSg1 @OBJ
"<.>"
        "." CLB <<<
\end{verbatim}
\caption{Analyse av den nordsamiske setninga \textit{Mun oainnán beatnaga nieidda uvssa duohkán.}}
\label{beatnaga}
\end{center}
\end{figure}
% vi har ikkje nokon hjelp i syntaktisk forstand, men med semantiske set, for eksempel human vs. not human, kan vi begrense mulighetan av hvem som er en potentiell eier
% En regel som sier REMOVE (@GN>) IF (NOT 0 HUMAN)(1 HUMAN); ville tatt bort ikkje-menneskelige eiere (som er den prototypiske genitiv-rollen) hvis det eide objektet er menneskelig.
% Problemer med akkurat denne regelen kunne være slike konstruksjoner som:
% Fornuftens mann.
% Fremtidens kvinne.
% I så tilfelle ville man kanskje måtte lage et set av nettopp de begrepan og ekskludere dem fra target-ordan av denne regelen.

Sjølv om det dreier seg om to grammatiske kasus med ulik posisjon i setningsstrukturen, har vi altså ofte tilfelle der vi ikkje finn rett svar. Eit autentisk døme er \ref{SDD1}. %%%

 \begin{example}\label{SDD1}
\gll SDD áigu deattuhit sámi álbmoga dárbbuid guovlluid dearvvašvuođafitnodagaid eaiggátstivremis.
      SHD vil vektlegge same.Gen folk.Gen behov.Gen områda.Gen helseforetaka.Gen eigarstyring.Loc
\glt  'SHD vil legge vekt på den samiske befolknings behov i eierstyringen av de regionale helseforetakene'
\glend
\end{example}

Her klarer ikkje programmet å gje noko analyse, vi får to analyser ståande att heile vegen. %%%

\begin{figure}[htbp]
\begin{center}
\begin{verbatim}
"<SDD>"
        "SDD" N ACR Sg Gen @GN>
"<áigu>"
        "áigut" V TV Ind Prs Sg3 @+FAUXV
"<deattuhit>"
        "deattuhit" V TV Inf @-FMAINV
"<sámi>"
        "sápmi" N Sg Gen @GN>
"<álbmoga>"
        "álbmot" N Sg Acc @OBJ
        "álbmot" N Sg Gen @GN>
"<dárbbuid>"
        "dárbu" N Pl Acc @OBJ
        "dárbu" N Pl Gen @GN>
"<guovlluid>"
        "guovlu" N Pl Acc @OBJ
        "guovlu" N Pl Gen @GN>
"<dearvvašvuođafitnodagaid>"
        "dearvvašvuođa#fitnodat" N Pl Gen @GN>
"<eaiggátstivremis>"
        "eaiggát#stivret" V* TV Actio N Sg Loc @ADVL
"<.>"
        "." CLB <<<
\end{verbatim}
\caption{Analyse av den nordsamiske setninga \textit{SDD áigu deattuhit sámi álbmoga dárbbuid guovlluid dearvvašvuođafitnodagaid eaiggátstivremis.}}
\label{SDD1-a}
\end{center}
\end{figure}


Ofte er desse døma prega av ein dominoeffekt, viss vi løyser den eine får vi også til den andre, som i \ref{SDD2}: %%%

\begin{example}\label{SDD2}
\gll SDD jođiha bajimus dási dearvvašvuođapolitihkalaš stivrema fitnodagaid hárrái.
      SDD styre øverste nivå helsepolitisk styring foretaka omsyn
\glt      SHD vil utøve en overordnet helsepolitisk styring med omsyn til foretakene.
\glend
\end{example}

Men framleis klarer programmet ingen av dei, jf. figur \ref{SDD2-a}, så her er det meir å gjere. %%%

\begin{figure}[htbp]
\begin{center}
\begin{verbatim}
"<SDD>"
        "SDD" N ACR Sg Nom @SUBJ
"<jođiha>"
        "jođihit" V TV Ind Prs Sg3 @+FMAINV
"<bajimus>"
        "bajit" A Superl Attr @AN>
"<dási>"
        "dássi" N Sg Acc @OBJ
        "dássi" N Sg Gen @GN>
"<dearvvašvuođapolitihkalaš>"
        "dearvvašvuođa#politihkka" N* laš A Attr @AN>
"<stivrema>"
        "stivret" V TV Actio Acc @OBJ
        "stivret" V TV Actio Gen @GN>
"<fitnodagaid>"
        "fitnodat" N Pl Gen @GP>
"<hárrái>"
        "hárrái" Po @ADVL
"<.>"
        "." CLB <<<
\end{verbatim}
\caption{Analyse av setninga \textit{SDD jođiha bajimus dási dearvvašvuođapolitihkalaš stivrema fitnodagaid hárrái}}
\label{SDD2-a}
\end{center}
\end{figure}



\subsection{Genitiv singularis / nominativ pluralis i lulesamisk}

I luleesamisk skil akkusativ og genitiv seg frå kvarandre (bortsett frå i dei personlege pronomena). I staden er genitiv singularis identisk med nominativ pluralis. Dette er ein homonymi som er langt enklare å disambiguere, i og med at det ikkje berre er kasusskilnad, men også numerusskilnad, og i ot med at både genitiv og nominativ er  grammatiske kasus som står i valensbunde tilhøve til kjerna i den frasen dei opptrer i. I eksempel \ref{bestefedre} har vi ei enkel setning med eit subjekt og eit finitt verb. Både subjektet og verbet er tvetydige (verbet kan vere presens tredje person pluralis eller preteritum andre person singularis, og substantivet altså genitiv singularis eller nominativ pluralis). %%%

\begin{example}\label{bestefedre}
\gll Ádjá mujttali
    Bestefar.GenPl fortel.Pl3
\glt 'Bestefedrene fortel'    
\glend
\end{example}%

Dei to orda i den lulesamisk setninga \ref{bestefedre} har altså begge to moglege analyser, som vist i figur \ref{am-multi}:

\begin{figure}[htbp]
\begin{center}
\begin{verbatim}
"<Ádjá>"
         "áddjá" N Pl Nom
         "áddjá" N Sg Gen
"<mujttali>"
         "mujttalit" V TV Ind Prs Pl3
         "mujttalit" V TV Ind Prt Sg2
"<.>"
         "." CLB
\end{verbatim}
\caption{Moglege morfologiske analyser i den lulesamiske setninga \textit{Ádjá mujttali}}
\label{am-multi}
\end{center}
\end{figure}

Setninga har likevel berre ein mogleg analyse, og den rudimentære lulesamiske disambigueraren vår finn denne analysen, jf. figur \ref{am}. %%%

\begin{figure}[htbp]
\begin{center}
\begin{verbatim}
"<Ádjá>" S:1418
        "áddjá" N Pl Nom
"<mujttali>" S:1511
        "mujttalit" V TV Ind Prs Pl3
"<.>"
\end{verbatim}
\caption{Analyse av den lulesamiske setninga \textit{Ádjá mujttali}}
\label{am}
\end{center}
\end{figure}

Reglane står i \ref{am-regel}, den første seier at vi skal fjerne genitivlesinga viss nominativ pluralis er eit alternativ, viss det er eit verb i tredje person pluralis til høgre, og det ikkje er førstepersonspronomen som kan gje  oss andre analyser til venstre. Regelen for Pl3 leitar etter sikre kandidatar for nominativ pluralis til venstre ($C$-en i $*-1C$ står for \textit{careful mode}, dvs. vi er forsiktig og krev at substantivet allereie skal vere disambiguert, og det er det, reglane i \ref{am-regel} er \textbf{ordna}, og står i eit feeding-bleeding-tilhøve til kvarandre: Med motsett rekkjefølgje ville vi ikkje ha fått tilslag. %%%

\begin{example}\label{am-regel}
\item[(a)] REMOVE (N Sg Gen) IF (*-1 BOS OR CS BARRIER (Pron Pers Du1))(0 (N Pl Nom))(1 (V Ind Pl3) LINK NOT 0 Imprt OR ImprtII);
\item[(b)] SELECT Pl3 IF (*-1C (N Pl Nom) BARRIER (Pron Pers Sg2 Nom) OR  (Pron Pers Sg3 Nom) OR (N Sg Nom));
\end{example}



\begin{example}\label{engel}
\gll Härrá ieŋŋgil Josefij niegon bihkusij ja javlaj:
    Herrens engel Josef.Ill draum.Ine {viste seg} og sa:
\glt 'Herrens engel viste seg til Josef i draumen og sa:' 
\glend    
\end{example}

Vi har komme langt kortare med lulesamisk enn med nordsamisk, så her har vi ingen ferdige reglar, og i og med at skriftspråket ikkje skil mellom sterkt og svakt stadiom for \textit{härrá}, har vi tre kandidatar som skal disambiguerast, som de ser i figur \ref{engelanalyse}. %%%

\begin{figure}[htbp]
\begin{center}
\begin{verbatim}
"<Härrá>"
        "hærrá" N Sg Gen
        "hærrá" N Pl Nom
        "hærrá" N Sg Nom
"<ieŋŋgil>"
        "ieññgil" N Sg Nom
"<Josefij>"
        "Josef" N Prop Mal Pl Com
        "Josef" N Prop Mal Sg Ill
        "Josef" N Prop Mal Pl Gen
"<niegon>"
        "niehko" N Sg Ine
"<bihkusij>"
        "bihkusit" V IV Ind Prt Sg3
        "bigos" A Pl Gen
        "bigos" A Sg Ill
        "bigos" A Pl Com
"<ja>"
        "ja" CC
"<javlaj>" 
        "javllat" V TV Ind Prt Sg3
"<:>"
        ":" CLB <<<
\end{verbatim}
\caption{Analyse av \textit{Härrá ieŋŋgil Josefij niegon bihkusij ja javlaj:}}
\label{engelanalyse}
\end{center}
\end{figure}

Disambigueringsregelen for denne og tilsvarande setningar må ta uthgangspunkt i dei finitte verba: \textit{javlaj} er utvetydig verb i tredje person eintal, og vi kan velje rett analyse for \textit{bihkusij} på grunnlag av det. Deretter treng vi eit subjekt til desse verba, og \textit{ieŋŋgil} har ingen andre analyser. Når vi er så langt kan vi konkludere med at genitiv er den beste analysa for \textit{Härrá}. Val av rett adverbialkasus for Josef tar utgangspunkt i at \textit{Josef} er eit eigennamn og dermed sannsynlegvis i eintal, ergo er det illativ som er den beste løysinga. %%%




\section{Disambiguere homonymi som opptrer i same grammatiske kontekst}

% LINDA 
% Døme på loccom der vi har eit leksikalsk definert sett av verb eller substantiv (head) (lo-verb/head vs. com-verb/head)(semantiske eigenskapar ved substantivet)
% Døme på loccom der vi har ulik grammatisk kontekst (lokativ-setjingar for eksempel habeo-konstruksjoner, partitivkonstruksjoner, koordinasjon, ...) mus lea X
% Døme på loccom der vi ikkje har nokon av delane (ting vi ikkje får til - vi får jo til :-) )

% Vise korleis vi bruker sett til dette.

Typiske døme på homonymi i same kontekst er adverbial, som er laust knytt til setninga, slik at fleire analyser er moglege. I nordsamisk er det komitativ singularis / lokativ pluralis som er det beste dømet på homonymi i same grammatiske kontekst. %%%





\subsection{Lokativ pluralis vs. komitativ singularis}

% Døme på loccom der vi har eit leksikalsk definert sett av verb eller substantiv (head) (lo-verb/head vs. com-verb/head)(semantiske eigenskapar ved substantivet) %%%

Lokativ pluralis og komitativ singularis er alltid identiske i nordsamisk, uavhengig av variasjon i stammeklasse og bøyingsklasse. Begge kasusa er adverbiale kasus, og i dei fleste tilfella kan begge to i utgangspunktet opptre i same setning. Sjølv om den eine kasusforma kan vere valensbunde, kan den andre i teorien også opptre. Det kan vere vanskeleg å avgjere om noko skal gjerast \textit{med båten} (komitativ singularis) eller \textit{i båtane} (lokativ pluralis). %%%

% Det er nødvendig å vite litt om bruken av lokativ og komitativ i nordsamisk.
% Mens lokativen er en utrolig flertydig kasus med mange funksjoner, er komitativen heller trang i forhold til bruken.
% Lokativen er har prototipisk en stedsfunksjon som både tilsteds- (inessiv i finsk) og frasteds-kasus (elativ i finsk?), i tilegg står eieren i en habitiv-konstruksjon i lokativ og lokativen brukes i en del-av-relasjon

% Komitativen er en prototypisk "verkty"-kasus. "Verktyet" kan være både menneskelig eller ikkje menneskelig og får dermed litt anna betydning.
% Den får altså enten en instrumental eller sosiativ betydning. Den sosiative betydninga kan parafraseres som enten "i selskap av" eller "med hjelp av".
% Idiomatiske betydninger er ikkje tatt med her.
% Dermed skaper en del predikater (verb, substantiv etc.) en potentiell lokativkontekst, mens andre skaper en komitativkontekst

Vi har fleire strategiar for å angripe dette problemet. Den opplagte måten er å bruke subkategoriseringskriteria: Visse sett av verb tar (eller krever) heller lokativ enn komitativ, og omvendt. Det følgende settet er en liste av verbs som tar lokativ som argument i motsetning til adjunkt. I realitetet er det vanskelig å skille mellom argument og adjunkt, det varierer hvor sterk kravet for en viss kasus er. %%%

\begin{figure}[htbp]
\begin{center}
\begin{verbatim}
LIST LOC-VERB = "ávkkástallat" "ballat" "beassat" "beroštit" "biehttalit"
			 	"bihtit" "boahtit" "ceavzit" "čuoččut" "čuovvut" "dinet" 
			 	"dolkat" "eastadallat" "eastadit" "fuolahit" "fuollat" 
			 	"geargat" "heaitit" "ilbmat" "jearrat" "luohpat" "nohkkot"
			 	"spiehkastit" "váruhit" "veallát" "oassálastit" "válga#oassálastit";
			 	%eg har oppdatert lokativ-settet

LIST COM-VERB = "árvalit" "árvvohuššat" "ávkašuvvat" "bargat" "bártašuvvat"
 				"buohtastahttit" "deaivvadit" "háladit" "hilbošit" "humadit"
 				"joatkit" "leaikkastallat" "meannudit" "molssodit" "náitalit"
 				"riidalit" "ságastaddat" "ságastallat" "šiehtadit" ;
 				%eg har oppdatert comitativ-settet
\end{verbatim}
\caption{Sett av lokativ- og komitativverb}
\label{loccomverb}
\end{center}
\end{figure}

Regel \ref{comikkjehabeo} vel komitativ (og ekskluderer dermed lokativ) innafor ei heil- eller leddsetning viss det ikkje er ein potensiell habeo-konstruksjon i vegen (som da kunne kreve lokativ). %%%

\begin{example}\label{comikkjehabeo}
SELECT Com IF (0 Sg)(*1 COM-VERB BARRIER CS OR CP LINK 0 VERB)
	(NOT *1 COPULAS BARRIER VERB LINK *1 COM-VERB BARRIER NOT-ADV-PCLE 
	LINK 0 Inf);    
\end{example}

Eit slikt døme er \ref{jordbruket}, der komitativen \emph{eanadoaluin} blir disambiguert av regelen \ref{comikkjehabeo}, jf. resultatet i figur \ref{comverbana}.%%%

\begin{example}\label{jordbruket}
\gll ...go sápmelaččat duođas álge eanadoaluin bargat.
     då samane {på alvor} byrja jordbruk.Com {å arbeide}
\glt 'Då samane på alvor byrja å arbeide i jordbruket'
\glend     
\end{example}

\begin{figure}[htbp]
\begin{center}
\begin{verbatim}
"<...>"
        "..." CLB <<<
"<go>"
        "go" CS @CS
"<sápmelaččat>"
        "sápmelaš" N Pl Nom @SUBJ
"<duođas>"
        "duođas" Adv @ADVL
"<álge>"
        "álgit" V IV Ind Prt Pl3 @+FMAINV
"<eanadoaluin>"
        "eana#doallu" N Sg Com @ADVL
"<bargat>"
        "bargat" V TV Inf @-FMAINV
"<.>"
        "." CLB <<<
\end{verbatim}
\caption{Analyse av \textit{...go sápmelaččat duođas álge eanadoaluin bargat}}
\label{comverbana}
\end{center}
\end{figure}



Disambiguering av komitativ og lokativ blir verre med komplekse setningar. For lange tekstar omsett frå norsk, som handlar om juridiske spørsmål er det vanskeleg å formalisere skilje mellom lokativ og komitativ. Norsk med-setning blir omsett med komitativ til samisk sjølv om det kanskje ikkje er den intuitive løysinga for samisk i utgangspunktet. %%%

%Korte setningar er enklare å disambiguere i ein samtalesituasjon, der vi har både semantikken til ordet (som ofte hjelper med å disambiguere mellom pluralis og singularis i det minste i nokre kontekstar), og med ein samtalapartner som er med og dannar kontekst og kontinuitet i samtalen, slik at setningan kjem i en viss kontekst. Hvis eg snakker om kosjn eg liker kaffe. Så er det meir sannsynlig at eg vil ha kvit kaffe og seier ''med mjølk'', enn at eg sier ''i mjølkene.''. Lausrive frå ein slik samtalesituasjon er det mogleg å tenkje seg fleirtal: Ei undersøking kan t.d. vise at det finst spor av aluminium i melkekartongane. - Kor har dei funne aluminium? - Jo, kan vi lese, i mjølkene (dvs. mjølkekartongane). %%%



\subsection{Singularis versus pluralis}

I visse tilfelle  går det an å disambiguere mellom lokativ og komitativ med å utnytte at den eine står i singularis og den andre i pluralis. %Også samisk skil mellom teljelege ord og uteljelege ord, sjølv om dette skiljet ikkje er så . Ein del pluralis-substantiv som for eksempel \emph{vuovttat} (hår) og \emph{sabehat} (ski) kan også brukast i singularis, men med ein litt forskjellig semantikk. Det betyr at konteksten vanligvis spiller ei rolle likevel. Ekte pluralisord, som alltid må stå i pluralis, er til dømes namn på festdagar, slik som \emph{beassažat} (Påske). %%%
Det fins også ord som ut ifra semantikken ikkje har ei pluralisform viss den ikkje blir framheva eksplisitt med hjelp av ord som \textit{máŋgalágan}, eit numeral eller eit ordenstal. Eigennamn er også substantiv som vanligvis ikkje fins i pl. Andre singularisord kan vere ord som allerede er spesifisert som unik. Namn på språk står t.d. i eintal, og får dermed komitativ. Andre slike singularisord kan være abstrakte ord som \emph{ipmárdus} og \emph{gelbbolašvuohta}. %%%

Dei andre tilfella der det er mulig å disambiguere singularis og pluralis går på konteksten. Nokre gjenstandar, objekt, abstrakte ting er unike i visse samanhengar, men ikkje i andre. Embetet "utenriksminister" finst det berre eitt av i Norge, men flere i Europa. Tilsvarande gjeld for fylkesting innafor kvart einskild fylke. Jf. setninga i \ref{fylke}, regelen er vist i \ref{fylkeregel} og den resulterande analysen i \ref{fylkeanalyse} %%%

\begin{example}\label{fylke}
Son deattuha sakka gulahallama sihke Sámedikkiin ja Finnmárkku fylkkadikkiin ášši gieđahaladettiin.
\end{example}

\begin{example}\label{fylkeregel}
SELECT Sg IF (-1 FYLKA LINK 0 (@GN>))(0 ("fylkka\#diggi" Com) OR ("fylkka\#gielda" Com) OR ("fylkka\#mánni" Com) OR ("fylkkas\#gielda" Com) OR ("fylkkarehket\#dárkun" Com))(NOT -3 FYLKA LINK -2 CC);
\end{example}

\begin{figure}[htbp]
\begin{center}
\begin{verbatim}
"<Son>"
        "son" Pron Pers Sg3 Nom @SUBJ
"<deattuha>"
        "deattuhit" V TV Ind Prs Sg3 @+FMAINV
"<sakka>"
        "sakka" Adv @ADVL
"<gulahallama>"
        "gullat" V TV halla Actio Acc @OBJ
        "gulahallat" V TV Actio Acc @OBJ
        "gullat" V* TV halla Actio N Sg Acc @OBJ
        "gulahallat" V* TV Actio N Sg Acc @OBJ
"<sihke>"
        "sihke" Adv @ADVL
"<Sámedikkiin>"
        "Sáme^diggi" N Prop Org Sg Com @ADVL
"<ja>"
        "ja" CC @CC
"<Finnmárkku>"
        "Finn^márku" N Prop Plc Sg Gen @GN>
"<fylkkadikkiin>"
        "fylkka#diggi" N Sg Com @ADVL
"<ášši>"
        "ášši" N Sg Gen @-FSUBJ
"<gieđahaladettiin>"
        "gieđahallat" V TV Ger @ADVL
"<.>"
        "." CLB <<<
\end{verbatim}

\caption{Analyse av \emph{Son deattuha sakka gulahallama sihke Sámedikkiin ja Finnmárkku fylkkadikkiin ášši gieđahaladettiin.}}
\label{fylkeanalyse}
\end{center}
\end{figure}

Visse ord, som abstrakte ord, opptrer sjelden i fleirtal. Vi testar ut ulike måtar å utnytte dette på, m.a. med settet i \ref{sgwds}. %%%


\begin{figure}[htbp]
\begin{center}
\begin{verbatim}
LIST SG-WORD = "ipmárdus" "doaivu" "dáro#giella" "gelbbolašvuohta"
 				("guovtte#gielalaš" Der/vuohta) "kultur#duogáš" 
 				"kultur#gelbbolašvuohta" "ovttas#bargu" "sáme#giella" ;
 				% oppdatert
 				% Ordan opptrer i flertall likevel når pluraliteten er understrekt ved hjelp av numeraler eller et av følgende ordan "goappeš", "goappašat", "earálágan", "máŋgalágan".
\end{verbatim}
\caption{Sett for ord som skal vere i singularis}
\label{sgwds}
\end{center}
\end{figure}

Ein av dei reglane  som viser til dette settet er \ref{sgregel}: %%%

\begin{example}\label{sgregel}
SELECT Sg IF (0 SG-WORD)(NOT *-1 Num OR Ord BARRIER NOT-ADJ)(*-1 BOS LINK NOT *1 ("goappeš") OR ("goappašat") OR ("earálágan") OR ("máŋgalágan"));
\end{example}

Setninga, eller snarare tittelen, i \ref{kvali}, kan sjå ut til å ha ekte ambiguitet i forhold til komitativ og lokativ: %%%

\begin{example}\label{kvali}
\gll Buoret kvalitehtta sámi giella- ja kulturgelbbolašvuođain
      Betre kvalitet samisk språk- og kulturkompetanse.Com
\glt 'Betre kvaliteten med hjelp av / i samisk språk- pg kulturkompetanse(ar)'
\glend
\end{example}


Det som til slutt gjør at berre den andre versjonen (komitativ) er mogleg, er  at kulturkompetanse bare fins i singular. Analysen blir dermed som i figur \ref{kvalianalyse}.%%%


\begin{figure}[htbp]
\begin{center}
\begin{verbatim}
"<Buoret>"
        "buoredit" V TV Imprt Prs Sg2 @+FMAINV
"<kvalitehtta>"
        "kvalitehtta" N Sg Nom @SUBJ
"<sámi>"
        "sápmi" N Sg Gen @GN>
"<giella->"
        "giella" N SgNomCmp Cmpnd @CMPND
"<ja>"
        "ja" CC @CC
"<kulturgelbbolašvuođain>"
        "kultur#gelbbolašvuohta" N Sg Com @ADVL
\end{verbatim}
\caption{Analyse av \textit{Buoret kvalitehtta sámi giella- ja kulturgelbbolašvuođain}}
\label{kvalianalyse}
\end{center}
\end{figure}


\section{Status quo}

Vi har sett at i arbeidet med å skilje mellom ulike adverbiale kasus tar vi mange ulike verkemiddel i bruk. Vi ser på dei leksikalske eigenskapane til verbet eller substantivet det fleirtydige ordet står til, og vi ser på leksikalske eigenskaper ved ordet sjølv. I tillegg utnyttar vi at formene skil seg frå kvarandre ikkje berre i kasus men også i numerus. %%%

Sjølv om lulesamisk har mindre kasushomonymi enn nordsamisk, har det også homonymi mellom genitiv og komitativ fleirtal, som i \textit{gielaj} i tabell \ref{smjcas}. I dette tilfellet skil kasusa seg meir frå kvarandre enn for den nordsamisk komitativ-lokativ-homonymien, i og med at genitiv er eit grammatisk kasus og komitativ er eit adverbialt. På den andre sida har dei same numerus, noko som gjer det vanskelegare att. %%%

Per oktober 2006 inneheld disambigueringsprogrammet vårt x reglar for tilordning av syntaktiske funksjonar, og y reglar for morfologisk disambiguering. Dei fjernar ikkje all homonymi, men etter at desse reglane har verka er biletet over homonymi eit heilt anna enn før disambigueringa.

% Tabeallat
%   \begin{table}[htdp]
%   \caption{Dei homonymiklassene i lulesamisk løpande tekst som står att etter disambiguering}
%   \begin{center}
%   \begin{tabular}{|c|r|l|}
%   \hline
%   \% hom & # hom & homonymiklasser \\
%   \hline
%   4,21\% & 6125 & N Sg Gen  =  N Pl Nom  \\ % etc.
%   \hline
%   \end{tabular}
%   \end{center}
%   \label{smjdishom}
%   \end{table}%

% Kommentárat.

% 

\section{Konklusjon}


I dette foredraget har vi vist at det er mogleg å disambiguere morfologisk homonymi med å sjå på den grammatiske og leksikalske konteksten orda står i. Vi har brukty føringsgrammatikk, ein grammatiske basert nedanfrå-og-opp-modell, heller enn alternativa, statistisk baserte nedanfrå-og-opp-modellar og grammatiske baserte ovanfrå-og-ned-modellar. Vi har vist at det er mogleg å få like gode resultat som for betre studerte språk. Vi har også vist at det er mogleg å oppno gode resultat for samisk disambiguering med ein modell som baserer seg på relativ posisjon i den lineæare strengen, trass i at samisk trandisjonelt har vorte rekna for eit språk med fri ordstilling. Våre resultat tyder på at i alle fall relevante delar av den samiske syntaksen har ei relativt fast ordstilling.  %%%

Sjølv om det er mogleg å formalisere konteksten grammatisk, er det mange tilfelle der grammatisk kontekst ikkje er nok til å disambiguere, som tilhøyrarar er vi avhengige av å forstå setninga for å velje rett analyse. I forsøka våre på å formulere denne forståinga på ein maskinlesbar måte, har vi konstruert sett av verb og substantiv med lik semantikk. Med desse relativt enkle tildaka har vi fått ein langt betre analyse. Vidare framskritt vil vere avhengig av ein semantisk analyse av heile leksikonet. I mange tilfelle vil også diskursen vere viktig for disambiguering. Den språklege diskursen kan vi ta med, ved å ta omsyn til større delar av teksten. Den utomspråklege diskursen kan vi til ein viss grad ta omsyn til, ved t.d. å gje programmet informasjon om tekstsjanger. %%%

I den grad resultata våre har overføringsverdi til teoriar om språkleg kompetanse, ser vi at den grammatiske analysen tilhøyraren gjev av setninga han eller ho høyrer, er avhengig av den leksikalske semantikken som er knytt til kvart einskild ord i setninga. %%%



\bibliography{lgtech}
%\bibliographystyle{alpha}
\end{document}


