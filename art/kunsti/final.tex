\documentclass[landscape,english,11pt]{seminar} 

 
\def\everyslide{\sf}
\usepackage{babel}
\usepackage{ucs}
\usepackage[utf8]{inputenc}

\usepackage[T1]{fontenc}

\usepackage{hyperref}
\usepackage{graphics}
%\slidesmag{5}
\slideframe{none}

%\usepackage{pp4slide}
%\usepackage{pause}


\title{Sámi language technology}
\author{Lene Antonsen, 
Saara Huhmarniemi, 
Marit Julien, 
Ilona Kivinen, 
Trond Trosterud, 
Linda Wiechetek} 
% \scalebox{0.30}[0.30]{\includegraphics{logoWeb070.jpg}}}

%\textit{http://giellatekno.uit.no} }\\ 
\begin{document}
\begin{slide}

\maketitle

\newslide
\textbf{Status quo: We have}
\begin{itemize}
\item finite-state transducers for North and Lule Sámi, partially for South Sámi
\item constraint grammars for North and partially for Lule Sámi
\item a corpus of appr 4 million words, and parallel texts in Norwegian for a small part of it
\item a file infrastructure and documentation, ready to be ported to other languages
\item For our sister project: A normative transducer to generate North and Lule Sámi spellers
\end{itemize}


\newslide
\textbf{Performance}
\begin{itemize}
\item The morphological transducer
\item The disambiguator
\begin{itemize}
\item Precision: 97.5 \%
\item Recall: 94.4 \%
\end{itemize}
\end{itemize}

\newslide
\textbf{Here: Two topics}
\begin{enumerate}
\item Where to focus to improve upon what we have
\item Where to go next
\end{enumerate}


\newslide
\textbf{Where to focus to improve upon what we have}
\begin{itemize}
\item The morphological transducer
\item The constraint grammar disambiguator
\item Pre- and postprosessing
\end{itemize}



\newslide
\textbf{The morphological transducer}
\begin{itemize}
\item Look at weak spots in the transducer
\begin{itemize}
\item Numerals
\item ...
\end{itemize}
\item make normative generators
\begin{itemize}
\item Only generate one form
\item Make different generators for different dialects
\end{itemize}
\item ...
\end{itemize}


\newslide
\textbf{The constraint grammar disambiguator}
CG disambiguation is the most accurate method available In spite of that, there is little research available on optimizing CG rule sets
\begin{itemize}
\item Rule evaluation
\item Rule ordering
\item Rule development
\item The role of semantics
\item Recall
\item CG on larger domains than the sentence
\end{itemize}


\newslide
\textbf{Pre- and postprosessing}
\begin{itemize}
\item Removing the sets out of the disambiguator
\item


\end{itemize}
\newslide
\textbf{Rule evaluation}
\begin{itemize}
\item 1/4 of our CG rules were never used when analysing our corpus
\begin{itemize}
\item Most of them are probably superseeded by earlier, more general rules
\item We would like to 
\end{itemize}
\item
\end{itemize}

\end{itemize}
\newslide
\textbf{Rule ordering}
\begin{itemize}
\item One may think of two principles
\begin{itemize}
\item Efficient rules early (disambiguate as much as possible as early as possible)
\item Safe rules early (postpone hard question until as many safe rules as possible have hit)
\item 
\end{itemize}
\item We would like to evaluate these, and to test with different rule order mechanisms
\item One possibility could be to leave some ordering to the computer
\end{itemize}

\end{itemize}
\newslide
\textbf{Rule development}
\begin{itemize}
\item From what we see, CG rules are written by starting out from the obvious cases, and gradually expand into harder ones
\item We would like to investigate an alternative method
\item Start with the complex cases
\begin{itemize}
\item Most of them are probably superseeded by earlier, more general rules
\item We would like to 
\end{itemize}
\item Investigate \textit{barriers}
\begin{itemize}
\item CG is, in essence, a one-dimensional simulation of a two-dimensional space
\item The program scans to the left and right for context condition fulfilment
\item But the relevant context consists of phrases, rather than flat strings
\item We have built sets of barriers, and complementary sets
\end{itemize}
\end{itemize}







\newslide
\textbf{Barriers}
\begin{itemize}
\item NP barriers
\begin{itemize}
\item Rule ordering
\item 
\end{itemize}
\item V-complex barriers
\item S barriers
\end{itemize}



\newslide
\textbf{The role of semantics}
  \begin{itemize}
  \item We would like to tag our lexica for semantic features
  \item We would like to move our sets out of the CG file, and tag items before they reach the CG
  \end{itemize}
\item Sets
 \begin{itemize}
 \item We need to develop our semantic sets further
 \end{itemize}

\newslide
\textbf{Recall}

\begin{itemize}
  \item We suffer under a temporary defect in the CG compiler: Contrary to CG-2, vislcg is not able to run in debug mode
  \item vislcg3 is reported to have fixed this bug
  \item With such a compiler in place, we would like to work with a Correct corpus
  \end{itemize}
\end{itemize}


\newslide
\textbf{CG on larger domains than the sentence}

The present CG analyser has the sentence as its disambiguation domain. We would like to try to disambiguate global dependencies by using a paragraph-domain CG analyser

\begin{itemize}
\item long distance bound anaphora, 
\item pronoun resolution 
\item pro-drop verb-forms
\end{itemize}

\newslide
\textbf{}
\begin{itemize}
\item
\end{itemize}


\newslide
\textbf{Where to go next}
\begin{itemize}
\item Concrete plans
\begin{itemize}
\item Pedagogical programs
\item Greenlandic sentence disambiguation
\end{itemize}
\item Our wishlist
\begin{itemize}
\item More Sámi languages
\item Text-to-speech
\item Dictionary integration
\item Machine translation
\end{itemize}
\end{itemize}


\newslide
\textbf{Pedagogical programs}
\begin{itemize}
\item Make more sentences for the visl interface
\item Make an interactive dialogue program
\item This requires:
\begin{itemize}
\item A constrained generator (only one correct form as output)
\item A flexible setup for dialogues
\item An open interface
\end{itemize}
\end{itemize}

\newslide
\textbf{Greenlandic sentence disambiguation}
\begin{itemize}
\item In cooperation with Oqaasileriffik, funded by a Nordic grant
\item An alpha version of the CG is already up and running
\item We will try to implement a better approach to CG rule writing
\end{itemize}

\newslide
\textbf{More Sámi languages}
\begin{itemize}
\item South Sámi
\begin{itemize}
\item We will get access to a dictionary, probably this year
\item The norm is in a state of flux (waiting is ok), but the speller program needs to make a speller
\item 
\end{itemize}
\item Inari and Skolt Sámi
\begin{itemize}
\item We will not do anything with these languages unless we get partners from Finland
\end{itemize}
\item Kildin Sámi
\begin{itemize}
\item The dictionary is in principle available in electronic format, albeit in an ugly and partly broken xml
\item There is a research center in Berlin wanting to cooperate with us on Kildin Sámi
\item The problems with the Kildin Sámi alphabet are solved
\end{itemize}
\end{itemize}

\newslide
\textbf{Text-to-speech}
\begin{itemize}
\item There is a research group wanting to do the speech part with us
\item We have a plan for making the text-to-phonetic-representation part
\item This is thus awaiting funding
\end{itemize}

\newslide
\textbf{Dictionary integration}
\begin{itemize}
\item By this we mean:
\begin{itemize}
\item Have a dictionary where the lexemes on each side are given continuation lexica
\item These lexica feed into an analyser/generator
\end{itemize}
\item This is the cornerstone for both intelligent dictionaries and machine translation
\item Actual language pairs are
\begin{itemize}
\item North Sámi vs. Finnish, Norwegian (bokmål, nynorsk), Swedish, Lule Sámi, English, German
\item Lule Sámi vs. Norwegian, Swedish
\end{itemize}

\end{itemize}

\newslide
\textbf{Machine translation}
\begin{itemize}
\item CG-based MT has a good record
\item A natural starting point for us is Finnish - North Sámi
\begin{itemize}
\item Direction \textbf{to} Sámi since the problem is too little Sámi text
\item From Finnish, since Finnish and Sámi are structurally very similar, and the best dictionary resources are between these languages
\end{itemize}
\end{itemize}

\newslide
\textbf{}
\begin{itemize}
\item
\end{itemize}

\newslide
\textbf{}
\begin{itemize}
\item
\end{itemize}

\newslide
\textbf{}
\begin{itemize}
\item
\end{itemize}

\newslide
\textbf{}
\begin{itemize}
\item
\end{itemize}




\end{slide}
\end{document}
