\documentclass[landscape,english,11pt]{seminar} 

 
\def\everyslide{\sf}
\usepackage{babel}
\usepackage{ucs}
\usepackage[utf8x]{inputenc}

\usepackage[T1]{fontenc}

\usepackage{hyperref}
\usepackage{graphics}
%\slidesmag{5}
\slideframe{none}

%\usepackage{pp4slide}
%\usepackage{pause}


\title{Where to go, and how?}
\author{Lene Antonsen, 
Saara Huhmarniemi, 
Marit Julien, \\
Ilona Kivinen, 
Trond Trosterud, 
Linda Wiechetek} 
% \scalebox{0.30}[0.30]{\includegraphics{logoWeb070.jpg}}}

%\textit{http://giellatekno.uit.no} }\\ 
\begin{document}
\begin{slide}

\maketitle


\newslide
\textbf{What we have (linguistic resources)}
\begin{itemize}
\item 
\item morphological transducers and constraint grammars for North and partially for Lule Sámi
\item a corpus of appr 4 million words, and parallel texts in Norwegian for appr 1 mill words
\item a file infrastructure and documentation, ready to be ported to other languages
\item For our sister project: A normative transducer to generate North and Lule Sámi spellers
\end{itemize}

\newslide
\textbf{What we have (positions)}
\begin{itemize}
\item One permanent, ordinary university position for Sámi language technology 
\item NOK 650000 for a pedagogical program (+)
\item Starting 2007-2009: 2,7 permanent positions for a UiT centre for Sámi language technology (name suggestions welcome)
\end{itemize}


\newslide
\textbf{The language society}
\begin{itemize}
\item 6 written lgs, 
\begin{itemize}
\item Russia: Kildin Sámi, (500)
\item Finland, Norway, Sweden: North Sámi (20000)
\item Finland: Inari and Skolt Sámi (á 2-300?)
\item Norway, Sweden: Lule Sámi (2000?), South Sámi (500?)
\end{itemize}
\item South Sámi has its most fluent speakers among small children
\item The number of North Sámi speakers will drop to 12000, and then steadily increase
\end{itemize}

\newslide
\textbf{Written languages}
\begin{itemize}
\item All orthographies 20-30 years old
\item Writing is a highly prised skill
\item Different principles behind the orthographies:
\begin{itemize}
\item Kildin: Cyrillic
\item Inari, Skolt, North: «Czech»
\item Lule, South: «Polish»
\end{itemize}
\item The similar languages North/Lule and Kildin/Skolt are orthographically deeply divided
\end{itemize}

\newslide
\textbf{The languages}
\begin{itemize}
\item Uralic, 6-10 cases, possessive suffixes, infinite verb forms, separate dual
\item \textbf{Much} non-concatenative morphology, for almost all forms and stem classes textit{guolli - guliid, boahtit - boađán - bohte }
\item A mixture of Turkish and Icelandic
\item No tones, but quantity very important
\item Dependent upon grammatical processing for e.g. Information Retrieval 
\end{itemize}


\newslide
\textbf{Education}
\begin{itemize}
\item Bachelor students get good jobs as civil servants, hard to keep them in academia
\item Finland: Helped by national focus on Uralic studies, and the similarity with Finnish
\item Sweden: 3 languages of equal size, Swedish in a stronger position than Norwegian
\item Norway: Tromsø (some Sámi linguists), Kautokeino (aspiring at university status, focus on language education)
\end{itemize}

\newslide
\textbf{Resources}
\begin{itemize}
\item Speech data: Taped dialect samples
\item Relatively small and bad bilingual dictionaries (some not that bad)
\item A terminological database which is a mixture of dictionary and terms
\item Shallow pedagogical grammars
\item Texts are very hard to get at
\end{itemize}

\newslide
\textbf{A strong will to strengthen the languages}
\begin{itemize}
\item Speech would be nice
\item The speller will be greated with joy (and disappointment, for the ones who realise they will still have to know how to write themselves)
\item There is strong need for terminology
\end{itemize}


\newslide
\textbf
\begin{itemize}
\item What should we do to build a Sámi lg tech centre?
\begin{itemize}
\item What topics should we focus upon?
\item How to build Sámi language technology research (education, persons, projects)?
\item What to deliver to the speech communities, in what order
\end{itemize}


\newslide
\textbf{More specific questions for the audience}
\begin{enumerate}
\item How to invoke semantics into our disambiguator
\item How to build interactive pedagogical dialogue systems
\item How to set up a transfer lexicon for a MT system?
\end{enumerate}


\newslide
\textbf{Semantics in the disambiguator}
\begin{enumerate}
\item Should we encode semantic categories (which ones) in the lexicon?
\item Should we make sets?
\item Should we make hyponym/hyperonym relations
\end{enumerate}




\newslide
\textbf{The pedagogical dialogue systems}
\begin{itemize}
\item Background: Negation verb, 9 person-number combination
\item Goal: Make a relative open dialogue system:
\begin{itemize}
\item M: I am the Machine. Where are you now?
\item I am at the university.
\item Where are you going?
\item I am go
\end{itemize}
\end{itemize}


\newslide
\textbf{The role of semantics}
\begin{itemize}
 \item We would like to tag our lexica for semantic features
 \item We would like to move our sets out of the CG file, and tag items before they reach the CG
 \item Sets
  \begin{itemize}
  \item We need to develop our semantic sets further
  \end{itemize}
\end{itemize}


\newslide
\textbf{Pedagogical programs}
\begin{itemize}
\item Make more sentences for the visl interface
\item Make an interactive dialogue program
\item This requires:
\begin{itemize}
\item A constrained generator (only one correct form as output)
\item A flexible setup for dialogues
\item An open interface
\end{itemize}
\end{itemize}


\newslide
\textbf{Text-to-speech}
\begin{itemize}
\item If we want to go into work on speech technology, where should we start?
\item There is a research group wanting to do the speech part of text-to-speech with us
\item We have a plan for making the text-to-phonetic-representation part
\item Any advices?
\end{itemize}

\newslide
\textbf{Dictionary integration}
\begin{itemize}
\item By this we mean:
\begin{itemize}
\item Have a dictionary where the lexemes on each side are given continuation lexica
\item These lexica feed into an analyser/generator
\end{itemize}
\item This is the cornerstone for both intelligent dictionaries and machine translation
\item Actual language pairs are
\begin{itemize}
\item North Sámi vs. Finnish, Norwegian (bokmål, nynorsk), Swedish, Lule Sámi, English, German
\item Lule Sámi vs. Norwegian, Swedish
\end{itemize}

\end{itemize}

\newslide
\textbf{Machine translation}
\begin{itemize}
\item 
\item Natural starting points for us are Finnish -> North Sámi and North Sámi -> Lule Sámi
\begin{itemize}
\item Direction \textbf{to} Sámi since the problem is too little Sámi text
\item From Finnish, since Finnish and Sámi are structurally very similar, and the best dictionary resources are between these languages
\end{itemize}
\end{itemize}


\newslide
\textbf{Plan for bootstrapping a North Sámi - Lule Sámi transfer dictionary}
\begin{itemize}
\item 
\end{itemize}




\end{slide}
\end{document}
