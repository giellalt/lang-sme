\documentclass[a4paper,english]{article}
\usepackage{babel}
\usepackage{ucs}
\usepackage[utf8]{inputenc}
\usepackage[T1]{fontenc}
\usepackage{psfrag} % ??? needed?
\usepackage{a4wide}

\usepackage{hyperref}
\usepackage{graphics}
\usepackage{url} % representing urls
\usepackage{covington} % ling examples

\usepackage{natbib} % ling-style bibliogr
\bibpunct{(}{)}{;}{a}{,}{,}

\begin{document}

\title{Two-dimensionality in a one-dimensional framework}

\author{Lene Antonsen and Trond Trosterud\\ 
Faculty of the humanities,  University of Tromsø}


\maketitle

%\tableofcontents

\section{Introduction}

Basically, constraint grammar is a one-dimensional framework, where ambiguous readings of individual wordforms are disambiguated on the basis of the right and left context of the word string. Bearing in mind the important fact that actual sentences indeed do consist of such strings, we may claim CG a psycholinguistic reality: Like actual listeners, it deduces ambiguity based on context.

From a syntactic view, the string does not consist of words, but of phrases,

The standard way of coping with this discrepancy in the CG community is faithful to the bottom-up character of the whole framework: Start with the simple cases, and gradually build rules for more complicated ones.

While sensible enough as a procedural device, we would in principle opt for a different approach: Taking the phrasal nature of sentence structure as a starting point, and build modular sets simulating the behaviour of phrases. Relevant here are complementary sets and barriers.

\section{Simulating the NP}

In order to establish the argument structure of a sentence, we need to find the main verb and its argument NPs. NP detection implies scanning the string for head nouns, looking either for the head nouns themselves or their complementary sets. Our examples are drawn from our joint work on North Sámi.

\begin{example}\label{islands}
... V ... N ... N ...
\end{example}

We build the complementary sets in a nested fashion, first defining the set of possible pre-head NP members, \ref{PRE-NP-HEAD}. Therafter we define a complementary set $NPNH$, \ref{NPNH}.


\begin{example}\label{PRE-NP-HEAD}
SET PRE-NP-HEAD = (Prop Attr) | (Prop @PROP>) | (A Attr) | (Pron Pers Gen) | (N Gen) | Num | Cmpnd | CC | (Pron Dem) | (Pron Refl Gen) | (Pron Indef) | (PrfPrc @AN>) | (PrfPrc @PrcN>) | PrsPrc | (A Ord) ; \\
\end{example}
\begin{example}\label{NPNH}
SET NPNH = WORD - PRE-NP-HEAD | ABBR ; \\                 
\end{example}

\section{Barriers}

About our barriers.


\bibliography{pstbib}

\bibliographystyle{alpha}


\end{document}
