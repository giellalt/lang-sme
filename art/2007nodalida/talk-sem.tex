% This is a combined text-slide document.
% In order to get the slide document, write

\documentclass[landscape,english,11pt]{seminar} 
\def\everyslide{\sf}


\usepackage{babel}
\usepackage{ucs}
\usepackage[utf8]{inputenc}
\usepackage[T1]{fontenc}
\usepackage{psfrag} % ??? needed?
\usepackage{hyperref}
\usepackage{graphics}
\usepackage{url} % representing urls
\usepackage{covington} % ling examples


\slideframe{none}



\begin{document}
\begin{slide}

\title{Two-dimensionality in a one-dimensional framework}
\author{Trond Trosterud\\ 
Faculty of the humanities, \\
University of Tromsø \\
\scalebox{0.30}[0.30]{\includegraphics{logoWeb070.jpg}}
}


\newslide
\maketitle 

%\tableofcontents

\newslide
\section{Introduction}










\newslide
\section{Barriers}






\newslide
\begin{example}\label{barriers}
... X ... W ... B ... Z 
\end{example}


\newslide
\begin{example}\label{posi-barriers}
Positive barrier
\item[] ... IF (*1 TARGET BARRIER B)
\end{example}

\begin{example}\label{nega-barriers}
Negative barrier
\item[(a)] ... IF (NOT *1 TARGET BARRIER B)
\item[(b)] ... IF (*1 DOMAINLIMIT BARRIER TARGET)
\end{example}


\newslide
\subsection{Types of barriers}


\newslide
\begin{table}[htdp]
\caption{Barrier types, and what they block}
\begin{center}
\begin{tabular}{|l|c|c|c|c|c|c|}
%\begin{tabular}{|l|p{5cm}|p{5cm}|}
\hline
Example &  notAdv  & NPNH & N/Pron & notArg & SB\&Infi & SB  \\ \hline
CLB     &  yes     & yes  & no     & yes    & yes     & yes \\
Subjunc &  yes     & yes  & no     & yes    & yes     & yes \\ 
\hline
InfinV  &  yes     & yes  & no     & yes    & yes     & no  \\
Verb    &  yes     & yes  & no     & yes    & no      & no  \\
Pr/Po   &  yes     & yes  & no     & yes    & no      & no  \\
\hline
Arghead &  yes     & yes   & no     & no     & no      & no  \\
Argmod  &  yes     & no    & yes    & no     & no      & no  \\ 
\hline
\end{tabular}
\end{center}
\label{btypol}
\end{table}%




\newslide
\textbf{Barriers for special contexts}
\begin{example}
REMOVE (V PrfPtc) IF (*-1 S-BOUNDARY BARRIER BE/HAVE)(0 Ambiguous-PrfPtc);
\end{example}




\begin{example}
<T> SELECT A IF (*-1 DOM-BOUNDARY BARRIER A)(*1 DOM-BOUNDARY BARRIER A)
\end{example}


\newslide
\textbf{Combining the uniqueness and the domain barrier}
\begin{example}\label{mix}
SELECT Inf IF (*-1 modal-verb BARRIER S-BOUNDARY OR INF);
\end{example}


\newslide
\subsection{Barriers in the Sámi and Norwegian taggers}

\begin{example}
Number of barriers
\item[(a)] OBT: 71 distinct barriers (49 non-unique, 1064 total)
\item[(b)] SME: 293 distinct barriers (111 non-unique, 1956 total)
\end{ example}






\newslide
\begin{example}\label{barrex}
... S ... X ... V ... Y ... S ... 
\end{example}



\newslide
\begin{example}\label{s}
SET S-BOUNDARY  = (Pron Interr) | (Pron Rel) | ("muhto") | MO | (";") | (":") | ("-") | ("–") | CS ;	
\end{example}



\newslide
\begin{example}
MAP (@GN>) TARGET Gen IF (*-1 BOC BARRIER Pr)(NOT 0 TIME)(*1 N BARRIER NPNHA LINK NOT 0 Prop);
\end{example}
\begin{example}
SELECT Adv IF (*-1 VFIN BARRIER S-BOUNDARY LINK NOT 0 Neg)(0 ("hárve"));
\end{example}


\newslide
\begin{example}\label{sv}
SET SV-BOUNDARY = S-BOUNDARY | @-FMAINV | @+FMAINV ;
\end{example}




\newslide
\begin{example}\label{svx}
... S ... X ... V ... V' ... S ...
\end{example}

\newslide
\begin{example}\label{ped}
\begin{itemize}
\item[(a)] SELECT A IF *-1 B ;
\item[(b)] SELECT A IF *-1 B BARRIER C ;
\item[(c)] SELECT V-PL IF *-1 (N Nom) LINK *-1 CC LINK -1 (N Nom)) ;
\end{itemize}
\end{example}




\newslide
\subsection{Not complex barriers}



\newslide
\section{Simulating the NP}



\newslide
\textbf{Scanning for N heads}
\begin{example}\label{islands}
... V ... N ... N ...
\end{example}



\newslide
\textbf{Baltic Finnic and Sámi NP structure}
\begin{example}\label{NP}
\begin{itemize}
\item[(a)] (Det-Cx) (Gen*) (A-Cx*) N-Cx(-Px) [Baltic Finnic]
\item[(b)] (Det-cx) (Gen*) (A-Attr*) N-Cx(-Px) [Sámi]
\end{itemize}
\end{example}


\newslide
\begin{table}[htdp]
\caption{North Sámi case paradigm}
\begin{center}
\begin{tabular}{|l|c|c|}
\hline
Case  & Sg       & Pl                        \\ \hline
Nom    & dat ođđa nieida      & dat ođđa nieiddat           \\ \hline
Acc    & dan ođđa nieidda    & daid ođđa nieiddaid         \\ \hline
Gen    & dan ođđa nieidda    & daid ođđa nieiddaid         \\ \hline
Ill    & dan ođđa niidii   & daidda ođđa nieiddaide      \\ \hline
Loc    & dan ođđa nieiddas   & dain ođđa nieiddain          \\ \hline
Com    & dainna ođđa nieiddain & daid ođđa  nieiddaiguin     \\ \hline
Ess & \multicolumn{2}{c|}{danin ođđa nieidan}      \\ \hline
\end{tabular}
\end{center}
\label{smecas}
\end{table}%


\newslide
\begin{table}[htdp]
\caption{Partial Finnish case paradigm}
\begin{center}
\begin{tabular}{|l|c|c|}
\hline
Case  & Sg       & Pl                        \\ \hline
Nom    & se    uusi tyttö       & ne uudet tytöt      \\ \hline
Acc    & sen    uuden tytön     & ne uudet tytöt     \\ \hline
Gen    & sen    uuden tytön     & niiden uusien tyttöjen     \\ \hline
Ill    & siihen uuteen tyttöön  & niihin uusiin tyttöihin   \\ \hline
Ine    & siinä uudessa tytössä  & niissä uusissa tytöissä      \\ \hline
Com    &                        & niine  uusine tyttöineen     \\ \hline
Ess    & sinä uutena tyttönä    & niinä uusina tyttöinä      \\ \hline
\end{tabular}
\end{center}
\label{fincas}
\end{table}%




\newslide
\begin{example}\label{PRE-NP-HEAD}
SET PRE-NP-HEAD = (Prop Attr) | (A Attr) | ("buorre") | (Pron Pers Gen) | (N Gen) | Num | Cmpnd nieidda | (Pron Dem) | (Pron Refl Gen) | (PrfPrc @AN>) | (PrfPrc @PrcN>) | PrsPrc | (A Ord) ; \\
\end{example}
\begin{example}\label{NPNH}
SET NPNH = WORD - PRE-NP-HEAD | ABBR ; \\                 
\end{example}



\newslide
\section{Conclusion}





       



\end{document}

