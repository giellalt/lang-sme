\documentclass[a4paper,english]{article}
\usepackage{babel}
\usepackage{ucs}
\usepackage[utf8]{inputenc}
\usepackage[T1]{fontenc}
\usepackage{a4wide}

\begin{document}

\title{Twol at work}

\author{Sjur Moshagen\\The Sámi Parliament\\Norway
\and Pekka Sammallahti\\ Giellagas Institute\\University of Oulu
\and Trond Trosterud\\Faculty of the humanities\\University of Tromsø}


\maketitle

\tableofcontents

\section{Introduction}


\section{Pekka's intro}
Languages with ample morphophonological variation pose a problem for automatic analysis. A recapitulation of historical changes (or phonological rules) in two-level rules may be an elegant solution from the point of view of an overall grasp of the language in question but one soon runs into difficulties in dealing with products of analogical levellings and other exceptions, especially when text words consist of several morphemes each interacting with the other phonologically. One such language is Saami where word stems interacting with affixes can have over 20 phonological variants and derivational and inflectional morphemes interacting with word stems or each other more than five. 

After unsuccesfully trying different morphophonological rule approaches to the variation stemming from morpheme interaction an \textbf{indexed concatenation model} was devised. This model provides \textbf{phonological/graphemic morpheme variants} with indexes (or abstract features) according to their continuation categories, the sets of suffixes it prededes. In practice a phonological/graphemic variant of a word stem occurring before a certain set of suffixes receives an index different from that of another variant occurring before a different set of suffixes. 

Since the distribution of stem variants in relation to suffix sets vary from one stem type to another, a single phonological/graphemic variant of a word stem may belong to two or more morphophonological variants if a different stem has two or more phonological/graphemic variants before the same set of suffixes. Accordingly the suffixes are indexed according to their relations with morphophonological word stem variants. 

The interaction with the word stems\textit{giehta} ‘hand; arm’ and\textit{njunni} ‘nose’ with the nominative plural suffix \textit{­t} and the second person singular accusative possessive suffix \textit{­t} may serve as an example. Both suffixes call for \textbf{weak grade} in the stem consonant center (the consonants between a stressed and a stressless vowel): (\textit{njunni →}) \textit{njuni-t} ‘noses’ and (\textit{giehta} →)\textit{ gieđa-t} ‘hands/arms’. However, the two suffixes call for different stem vowel alternants in i-stems but not in a-stems: \textit{njuni-t} ‘noses’ ≠ \textit{njuná-t} ‘your nose’ but \textit{gieđa-t} ‘hands/arms’ = \textit{gieđa-t} ‘your hand/arm’. The stems \textit{njuni-} and \textit{njuná-} receive different indexes or abstract features (such as \textit{njuni-N1} and \textit{njuná-N2}) because they call for different sets of suffixes but so do the corresponding instances of the stem \textit{gieđa-} (\textit{gieđa-N1} and \textit{gieđa-N2}) because the suffixes it precedes belong to two sets. 

Since nouns have partly the same mophophonological variants as verbs, the morphophonological variants receive three kinds of indexes: N for the morphophonological variants of nouns, V for the morphophonological variants of verbs, and X for the morphophonological variants shared by nouns and verbs.

 

etc.


\section{Sjur's intro}


 
\bibliography{lgtech}
\bibliographystyle{alpha}


\end{document}