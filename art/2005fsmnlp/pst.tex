\documentclass[a4paper,english]{article}
\usepackage{babel}
\usepackage{ucs}
\usepackage[utf8]{inputenc}
\usepackage[T1]{fontenc}
\usepackage{psfrag}
\usepackage{a4wide}

\begin{document}

\title{Twol at work}

\author{Sjur Moshagen\\ The Sámi Parliament\\ Norway
\and Pekka Sammallahti\\ Giellagas Institute\\ University of Oulu
\and Trond Trosterud\\ Faculty of the humanities\\ University of Tromsø}


\maketitle

\tableofcontents

\section{Introduction}

In this article, we will show two-level morphology at work. In sections \ref{twol} and \ref{disamb}, we will lay out the foundation for the work, by presenting the philosophy behind the Northern Sámi two-level parser. In sections \ref{ped}, \ref{spell} and \ref{term} we then have a look at different applications, pedagogical programs, spell checking and terminology management.

\section{Two-level morphology}\label{twol}

Languages with ample morphophonological variation pose a problem for automatic analysis. A recapitulation of historical changes (or phonological rules) in two-level rules may be an elegant solution from the point of view of an overall grasp of the language in question but one soon runs into difficulties in dealing with products of analogical levellings and other exceptions, especially when text words consist of several morphemes each interacting with the other phonologically. One such language is Saami where word stems interacting with affixes can have over 20 phonological variants and derivational and inflectional morphemes interacting with word stems or each other more than five. 

After unsuccesfully trying different morphophonological rule approaches to the variation stemming from morpheme interaction an \textbf{indexed concatenation model} was devised. This model provides \textbf{phonological/graphemic morpheme variants} with indexes (or abstract features) according to their continuation categories, the sets of suffixes it prededes. In practice a phonological/graphemic variant of a word stem occurring before a certain set of suffixes receives an index different from that of another variant occurring before a different set of suffixes. 

Since the distribution of stem variants in relation to suffix sets vary from one stem type to another, a single phonological/graphemic variant of a word stem may belong to two or more morphophonological variants if a different stem has two or more phonological/graphemic variants before the same set of suffixes. Accordingly the suffixes are indexed according to their relations with morphophonological word stem variants. 

The interaction with the word stems\textit{giehta} ‘hand; arm’ and\textit{njunni} ‘nose’ with the nominative plural suffix \textit{­t} and the second person singular accusative possessive suffix \textit{­t} may serve as an example. Both suffixes call for \textbf{weak grade} in the stem consonant center (the consonants between a stressed and a stressless vowel): (\textit{njunni $\rightarrow$}) \textit{njuni-t} ‘noses’ and (\textit{giehta} $\rightarrow$)\textit{ gieđa-t} ‘hands/arms’. However, the two suffixes call for different stem vowel alternants in i-stems but not in a-stems: \textit{njuni-t} ‘noses’ ≠ \textit{njuná-t} ‘your nose’ but \textit{gieđa-t} ‘hands/arms’ = \textit{gieđa-t} ‘your hand/arm’. The stems \textit{njuni-} and \textit{njuná-} receive different indexes or abstract features (such as \textit{njuni-N1} and \textit{njuná-N2}) because they call for different sets of suffixes but so do the corresponding instances of the stem \textit{gieđa-} (\textit{gieđa-N1} and \textit{gieđa-N2}) because the suffixes it precedes belong to two sets. 

Since nouns have partly the same mophophonological variants as verbs, the morphophonological variants receive three kinds of indexes: N for the morphophonological variants of nouns, V for the morphophonological variants of verbs, and X for the morphophonological variants shared by nouns and verbs.

\section{Disambiguation}\label{disamb}

Disambiguation may be done in several ways. One is Finite-state intersection grammar, as suggested by \cite{Koskenniemi97}. In our Sámi project, we have chosen a different path, and use constraint grammar, as presented by \cite{Tapanainen96}, here in Eckhard Bick's open-source version \textit{vislcg}. This component is being written by Trond Trosterud and Marit Julien. Although still under development, it is already good enough to match the level of statistically-based POS taggers. At its present stage, it contains approximately 1300 disambiguation rules.


\section{Twol in use}

A grammatical analysator can be used for many purposes. We will here have a look at some areas where the Sámi analysator has been put to use.

\subsection{Pedagogical programs}\label{ped}

The Sámi analysator has been used to analyse sentences for interactive pedagogical syntax learning in the so-called visl project (Visual Syntax Learning) at Syddansk Universitet. The format in itself is not dependent upon having a grammatical analysator, but the analysator makes it possible to add sentences automatically. The process behind the analysis in Figure \ref{underlying} consists of three parts:

\begin{enumerate}
\item The morphological analyser gives all possible analyses
\item A morphological disambiguator removes the incorrect ones, and add syntactic functions
\item A phrase structure grammar gives the linear representation a hierarchical structure
\end{enumerate}

At present, beta versions of the first two components are in place. The third component is still missing.
For a sentence like \textit{Áhčči lea oastán munnje divrras sabehiid} 'Father has bought me a an expensive pair of skis', the morphological analyser gives the representation in Figure \ref{fatherall}.

\begin{figure}[htdp]
\caption{Morphological analysis}
\begin{center}
\begin{verbatim}
"<Áhčči>"
         "áhčči" N Sg Nom
"<lea>"
         "leat" V Ind Prs Sg3
"<oastán>"
         "oastit" V PrfPrc
         "oastit" V* N Actor Sg Nom PxSg1
         "oastit" V* N Actor Sg Gen PxSg1
         "oastit" V* N Actor Sg Acc PxSg1
         "oasti" N Sg Nom PxSg1
         "oasti" N Sg Gen PxSg1
         "oasti" N Sg Acc PxSg1
"<munnje>"
         "mun" Pron Pers Sg1 Ill
"<divrras>"
         "divrras" A Attr
         "divrras" A Sg Nom
"<sabehiid>"
         "sabet" N Pl Gen
         "sabet" N Pl Acc
"<.>"
         "." CLB
\end{verbatim}
\end{center}
\label{fatherall}
\end{figure}%

After disambiguation and adding of syntactic functions, the same sentence can be seen in Figure \ref{fatherdis}.

\begin{figure}[htdp]
\caption{Disambiguated version}
\begin{center}
\begin{verbatim}
"<Áhčči>"
        "áhčči" N Sg Nom @SUBJ
"<lea>"
        "leat" V Ind Prs Sg3 @+FAUXV
"<oastán>"
        "oastit" V PrfPrc @-FMAINV
"<munnje>"
        "mun" Pron Pers Sg1 Ill @ADVL
"<divrras>"
         "divrras" A Attr @AN>
"<sabehiid>"
        "sabet" N Pl Acc @OBJ
"<.>"
\end{verbatim}
\end{center}
\label{fatherdis}
\end{figure}%

The underlying representation of the pedagogical program takes the disambiguated analysis as input, and makes a tree structure, as seen in Figure \ref{underlying}. At present, this process is only partly automatised, the bracketing of constituents (denoted with '=') must be done manually, making such a component is the next task ahead.

\begin{figure}[htdp]
\caption{Underlying representation in the pedagogical program}
\begin{center}
\begin{verbatim}
S:n('áhčči',sg,nom)	Áhčči
P:g
=D:v('leat',ind,pr,3sg)	lea
=H:v('oastit',pcp2)	oastán
A:pron('mun',<pers>,1sg,ill)	munnje
Od:g
=D:adj('divrras',attr)  divrras
=H:n('sabet',pl,acc)	sabehiid
\end{verbatim}
\end{center}
\label{underlying}
\end{figure}%

A pilot set of 200 Sámi sentences will shortly be included in the \textit{http://visl.sdu.dk/visl/}. As an illutration of the format, figure \ref{fin} gives an impression of what the representation will look like, for the sentence \textit{Vanhemmat olivat pitkän riidan jälkeen kauhean väsyneitä} (The parents were terribly tired after a long argument). The sentence is taken from the Finnish sentences at the \textit{visl} homepage, and int is analysed by Esa Aaltonen.


\begin{figure}
\caption{Graphical version of a similar sentence in Finnish}
\begin{center}
\includegraphics{snt.png}
\end{center}
\label{fin}
\end{figure}


Later, when the analysator becomes better, it will also be possible to offer an open system, where the computer analyses user input and offers an interface for the user to analyse for himself. Today, this has been done for Danish.



\subsection{Spell checker}\label{spell}

Since October 2004 the Nowregian Saami Parliament has been running a project to create proofing tools for North and Julev (Lule) Saami. The project is based on the work by Pekka Sammallahti and the projects at the University of Tromsø.

ikkje-permissiv automat - korleis får vi det til?

\subsubsection{Non-permisive automaton and speller}

Twol som generator - generer fullformslister for språkteknologisk svake stavekontrollar

The project will create spellers and hyphenators for several applications on Windows, MacOS X and Linux. Spellers for applications like MS Office based on two-level technology have been described elsewhere several times (cf. xxx & yyy), and will not be discussed here. Instead we will discuss briefly how two-level technology can be applied to a speller engine like aSpell (aSpell-ref).

aSpell is a derivative of iSpell, and employs a simple one-level automata model for its engine. It supports in its latest version what is called "affix lexicons"[xxx???], which is a limited implementation of continuation lexicons.

xfst includes a command to generate a fullform list from non-recursive transducers. We have also marked the circular entries in the lexicon, making it easy to create a subset of it that is non-recursive. This model still contains a lot of word formation through derivation, but nothing based on compounding.

In addition  to the above, will we lexicalise all entries found in written material, such that

\subsection{Terminological database}\label{term}

Then something here.

Twol som generator 2: - genererer paradigmer for samiske termar i sametinget sin termbase

\section{Summing up}

liuhgiluewrhn.

 
\bibliography{lgtech}

\bibliographystyle{alpha}


\end{document}
