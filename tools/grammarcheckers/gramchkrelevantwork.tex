\documentclass[a4paper,english,12pt]{article}
\usepackage{babel}
\usepackage{ucs} %sami letters
% \usepackage{amssymb} %mathematical
\usepackage[utf8x]{inputenc}
\usepackage[T1]{fontenc}
\usepackage{harvard}

\usepackage[dvips]{graphicx}
\usepackage{rotating} 

%\usepackage{tikz}
%\usepackage{array}
%\usepackage{arydshln} %has to be after array
%\usepackage{multirow}
\usepackage{graphics}
\usepackage{graphicx}
\usepackage{tabularx} %specified width
\usepackage{tipa}
\usepackage{booktabs}
\usepackage{ctable} %loads booktable by default
\usepackage{colortbl}
\usepackage{covington}
\usepackage{url}
\usepackage{harvard}
\usepackage[right=2.5cm,left=2.5cm,top=2cm,bottom=2cm]{geometry}
% \usepackage{bibtexlogo}
\usepackage{setspace}

\usepackage{fancyhdr}
\usepackage{linguex}

\pagestyle{fancy}
%\fancyfoot[LO,LE]{\slshape Rule-based Machine Translation from North to Lule Sámi}


\begin{document}

\setcounter{secnumdepth}{3}
\setcounter{tocdepth}{3}
%\linespread{1.5}
\begin{spacing}{1.0}


\newcommand{\tx}{\mbox{t\hspace{-.35em}-}} % for S‡m
\newcommand{\txx}{\mbox{T\hspace{-.5em}-}} 




\title{{\Large Grammar checking - strategy}}


\author{Linda Wiechetek \\
		Universitetet i Tromsø \\
			(Norway)}
\date{}
\maketitle

\thispagestyle{empty}
\tableofcontents 
\thispagestyle{empty} %has to be after \maketitle

\newpage

\setcounter{page}{1} %in order to start pagenumbering

\parindent = 0mm
\parskip = 12pt


\section{Lingsoft's Grammarcheckers}

\begin{itemize}
\item Finnish (\url{http://www.lingsoft.fi/print.php?lang=en&doc_id=458})
\item Danish (\url{http://www.lingsoft.fi/dangrc/errors})
\item Swedish (\url{http://www.lingsoft.fi/swegrc/errors})
\item Norwegian (\url{http://www.lingsoft.fi/nobgrc/errors})
\end{itemize}

\subsection{Grammar errors treated}


\begin{itemize}
\item 40 error categories
\item distinguish between writing convention and grammar errors
\item 
\item 
\end{itemize}

\subsubsection{Writing convention errors}
\begin{itemize}
    \item Several spaces in a row
    \item Spaces in conjunction with quotation marks, parentheses, punctuation marks, special characters, such as %, §, © and °
	\item Correct punctuation (Hyphens, Dashes, Quotations, Parentheses, Multiple emphasis, Multiple punctuation marks, Ellipses 
	\item Parity of parentheses, brackets, braces
    \item Uneven number of quotation marks 
    \item Recognition and correction of certain abbreviations not handled by the spell checker
    \item Formatting of numbers (Phone numbers, Date expressions, long ordinal and decimal numbers)
    \item Sentence beginning with a lowercase letter
    \item Interrogative sentence beginning with a question word ending in full stop 
\end{itemize}

\subsubsection{Grammar errors (Finnish)}
\begin{itemize}
    \item No main clause
    \item No finite verb
    \item Number of finite verbs
    \item Sentence beginning with a coordinating conjunction 
	\item Agreement (subject-predicate, NP-internal number and case)
    \item Negative forms (periphrastic structures)
    \item Tense (periphrastic structures)
    \item Verb chains
    \item Double passive 
	\item Comparison: mitä/kaikkein + superlative, yhä + comparative, Comparative + kuin, mahdollisimman + positive, Not enemmän + positive, Not kaikista + superlative 
	\item Compounding (standardized compounds erroneously written as two words, standardized two word phrases erroneously compounded) 
 	\item Unnecessary repetition of clitics
    \item myös + -kin
    \item ja + ei
    \item Multiple sequential subordinate clauses
    \item Illegal endings attached to numbers
    \item Stylistically marked words
\end{itemize}

\subsubsection{Grammar errors (Swedish)}

\begin{itemize}
    \item Noun phrase (Definiteness form of noun/adjective)
    \item Number agreement: determiner and noun, adjective
    \item Gender agreement: determiner and noun, adjective and noun, pronoun and noun 
    \item Masculine form of adjective %    Note: This is a check for the masculine e-form of adjectives in front of unambiguous neuter nouns and some common (semantically) feminine nouns.
	\item	 Supine without "ha",    Double supine 
	\item    Double passive,    S-passive after certain verbs   
	\item  Infinitive after preposition, Infinitive without/with "att"
	\item    Number of finite verbs, No verb, No finite verb
	\item   Word order:  Position of adverb in subordinate clauses, negated element in subordinate clauses, Constituent order in subordinate interrogative clauses
	\item    Double negation
	\item     Subject complement agreement
	\item    Use of preposition with two-part conjunctions 
	\item    The construction ”möjligast” + adjective
\end{itemize}

\subsubsection{Grammar errors (Norwegian)}


\begin{itemize}
    \item     Definiteness form of noun conforming to definiteness of determiner
    \item Definiteness form of adjective when preceded by determiner and followed by noun
    \item Gender agreement: determiner and noun, adjective and noun when preceded by determiner
    \item Number agreement: determiner and noun, adjective and noun when preceded by determiner
    \item Infinitive without å
    \item Unambiguous infinitive requiring å
    \item Modal auxiliary verb agreement
    \item Past participle without ha
    \item Double s-passive
    \item S-passive after certain verbs
    \item Number of adjacent unambiguous finite verbs
    \item Lack of verb in sentence
	\item Word order: Position of clausal adverbs in subordinate clauses, Position of subject in simple clauses with initial adverb (inversion)
	\item Possessives in postposition: Definiteness form of noun, Gender agreement: noun and possessive, Number agreement: noun and possessive
    \item Agreement between subject and subject-complement: The check for non-agreeing subject complements is restricted to sentence-initial main clauses with the copula være
    \item Negative polarity items: Warning and suggestion mechanism regarding correct uses of ingen and noen as determiner in positive and negative sentences
    \item Gender agreement: quantifying pronoun and av-phrase
    \item Superfluous reflexive pronoun
    \item Pronoun form after preposition
\end{itemize}

\subsubsection{Grammar errors (Danish)}

\begin{itemize}
    \item Noun phrase: Definiteness: Double determination
    \item Definite form of noun after indeterminate determiners and plain adjectives
    \item Definite form of adjective after definite determiners
    \item Gender agreement: Determiner and noun, Adjective and noun, Pronoun and noun (certain pronouns like en/et, hver/hvert in front of an af + plural noun phrase)
    \item Number agreement: Determiner and noun, Adjective and noun
	\item    Infinitive with/without at
    \item Double -s passive
    \item Modal auxiliary verb agreement
    \item Past participle without have
    \item Supine vs. past participle after auxiliary verbs
    \item Number of adjacent unambiguous finite verbs
    \item Lack of verb in sentence
	\item Correct adjective gradation
	\item Word order: Position of clausal adverbs in subordinate clauses, Position of subject in simple clauses with clause initial adverb (inversion), Inversion in subordinate clauses, Postposed possessives/genitives, Position of negative objects
	\item Subject - subject complement agreement gender/number-wise in simple copula clauses
    \item Oblique form of personal pronouns when governed by a preposition
    \item nogen vs. nogle
    \item heller vs. hellere
\end{itemize}

\section{Other errors}

\begin{itemize}
\item 1) Errors caused by a sloppy use of the "cut and paste" options of modern text editors, and 2) errors due to people's mistaken beliefs about the written language norms. \citet{Hagen2001a}
\item word doubling
\item comma errors
\item Parity of parentheses, of brackets, of braces (FINGRC)
\item  Uneven number of quotation marks (FINGRC)
\item Errors caused by a sloppy use of the "cut and paste" options of modern text editors
\end{itemize}

\section{Other CG grammar checkers}

\begin{itemize}
\item Catalan system: makes use of CATCG, a modular general-purpose shallow morphosyntactic parser for unrestricted Catalan text (Badia et al., 2001), several error detection modules (CatSpel - dealing with non-words; CG-based morphological error detection module: simple orthographic errors resulting in words such as wrong use of apostrophe, or wrong NP-agreement; CG-based negative n-gram error detection module: wrong word sequences; CG-based morphosyntactic error detection module: subject-verb disagreement, subcategorization errors, etc.; domain-specific modules: certain semantic or pragmatic errors), CATCG’s morphological disambiguation grammar and syntactic mapping module has been modified in such a way that word sequences that otherwise would imply a wrong or simply a non disambiguation are now analysed as a possible reading.
%Up to now only bigrams and trigrams have been used (also implemented
%using CG-based grammar). They have been collected
%via a semi-automatic procedure of detection errors
%in a POS-tagged corpus similar to the one used by (Kveton
%and Oliva, 2002).
\item
\item
\end{itemize}

\section{Precision and Recall}

\begin{itemize}
\item Precision: good alarms divided by the sum of good alarms and false alarms
\item Swedish grammar checker (see Birn 2000, Arppe 2000): resulting precision 70\% (according to Birn 2000:38), counted as good alarms divided by the sum of good alarms and false alarms. 
\item Norwegian grammar checker is 75 \% precision from a test corpus of 890 000 words from the newspapers Nordlys and Sarbsborg Blad \citet{Hagen2001a} (However, depending on what kind of text that is used as a test corpus and what kind of rules that are included, these numbers can vary a great deal. For example, if we include a rule that tests whether there are any finite verbs in a sentence, and apply the grammar checker on newspaper text with a lot of verb-less headlines, the precision rises to approximately 91 \% (and the same goes for the Swedish grammar checker).
\end{itemize}







\end{document}